\section{Literature Review}
\label{sec:literature-review}

\subsection{Trajectory Anomaly Detection}
\label{sec:anomaly-review}

\subsubsection{Statistical and Traditional Methods}
\label{sec:statistical-traditional}

Statistical approaches demonstrate the essential properties that synthetic trajectory data must preserve to maintain utility for anomaly detection research. Different detection methods rely on fundamentally distinct trajectory characteristics, establishing specific requirements for data generation.

Distance-based methods like Wang et al.~\cite{wangStatisticalFrameworkTaxi2020} work by comparing route lengths and travel patterns against historical distributions. For synthetic data to support this type of research, it must maintain realistic distance distributions and route variation patterns. Similarly, density-based approaches such as He et al.~\cite{heEnhancedDBSCANMultiple2020} depend on preserving local neighborhood structures - how trajectories cluster together spatially affects detection performance significantly.

The most successful traditional method has been isolation-based detection, particularly Zhang et al.~\cite{zhangIBATDetectingAnomalous2011}'s iBAT algorithm. This approach groups trajectories by origin-destination pairs and converts routes into symbolic sequences of grid cells. This method establishes two critical requirements for synthetic data generation: preserving origin-destination flow patterns and maintaining consistent spatial traversal sequences between locations.

Traditional methods also demonstrate a significant research gap that synthetic data generation addresses. Most approaches struggle with parameter sensitivity and insufficient labeled anomaly data~\cite{zhangIBATDetectingAnomalous2011}, creating challenges for systematic evaluation of new detection algorithms. Synthetic generation provides a solution through controlled datasets with known anomaly labels and adjustable parameters for systematic evaluation.

\subsubsection{Deep Learning Approaches}
\label{sec:deep-learning}

Deep learning approaches present distinctive challenges for synthetic data generation, as these methods depend on learning complex patterns that traditional approaches cannot capture.

Autoencoder-based detection, exemplified by Huang et al.~\cite{huangLSTMAutoencodersAttention2021}'s LSTM-AE-Attention model, operates by learning to reconstruct normal trajectory patterns. Anomalous trajectories that exhibit poor reconstruction quality are identified as suspicious. This approach establishes a critical requirement for synthetic data: preservation of subtle temporal patterns and sequence dependencies that characterize real trajectories, as their absence would compromise reconstruction-based detection effectiveness. The study also identifies a practical challenge where real datasets exhibit significant imbalance, with approximately 12 normal trajectories for every anomalous one, complicating training processes.

\textbf{Language Model-based Trajectory Anomaly Detection.} A significant breakthrough in trajectory anomaly detection comes from the application of language modeling techniques to spatial-temporal sequences. Mbuya et al.~\cite{mbuyaTrajectoryAnomalyDetection2024} introduce LM-TAD, an autoregressive causal-attention model that treats trajectories as sequences of tokens, similar to language statements. This approach leverages the inherent similarities between language and trajectory data, where both consist of ordered elements requiring coherence through external rules and contextual variations.

LM-TAD learns probability distributions over trajectories using transformer architectures, enabling identification of anomalous locations through perplexity and surprisal rate metrics. The model incorporates user-specific tokens to account for individual behavior patterns, significantly enhancing anomaly detection tailored to user context. This approach demonstrates superior performance on both synthetic and real-world datasets, particularly excelling at detecting user-contextual anomalies on the Pattern of Life (PoL) dataset while achieving competitive results on taxi trajectory data.

The language model paradigm addresses several critical limitations of traditional deep learning approaches: it provides interpretable anomaly scores through perplexity metrics, supports online detection through attention state caching, and handles diverse trajectory representations including GPS coordinates, staypoints, and activity types. This versatility makes it particularly suitable for synthetic data generation applications, as it can work with various data formats while maintaining strong detection performance. The integration of LM-TAD with diffusion-based generation forms a core component of the methodology described in Section~\ref{sec:methodology}.

More recent work with diffusion models, such as Li et al.~\cite{liDiffusionModelsVehicle2023}'s DiffTAD, demonstrates that synthetic trajectory generation can be directly applied for anomaly detection. Their approach treats trajectory generation as a denoising process, achieving significantly superior performance compared to traditional methods. This suggests potential adaptation of synthetic data generation techniques developed for privacy protection to anomaly detection applications.

Deep learning methods present specific requirements for synthetic data research, requiring large training datasets and performing optimally when learning from diverse trajectory patterns. Synthetic data generation addresses these requirements by providing abundant, diverse trajectory data that preserves essential characteristics necessary for effective anomaly detection.

\subsubsection{Spatio-Temporal Pattern Analysis}
\label{sec:spatio-temporal}

Identifying the critical patterns in trajectory data defines the preservation requirements for synthetic generation. Research demonstrates that trajectories exhibit multi-level structural properties essential for anomaly detection algorithm effectiveness.

At the spatial level, Zhang et al.~\cite{zhangIBATDetectingAnomalous2011} demonstrate that converting continuous GPS traces into grid-based symbolic sequences achieves effective anomaly detection. This indicates that synthetic data need not precisely replicate individual GPS coordinates, but must preserve the sequence of spatial regions traversed by vehicles. Their approach effectively manages variable GPS sampling rates, which is significant given that synthetic data may exhibit different temporal characteristics than real data.

Temporal patterns exhibit greater complexity than initially apparent. Chen et al.~\cite{chenTemporalContextAwareRoute2021} demonstrate that normal behavior definitions vary significantly based on temporal context - routes considered normal during off-peak hours may appear highly suspicious during rush hour periods. This requires synthetic data generation to preserve time-dependent behavioral patterns in addition to spatial accuracy.

Large-scale analysis provides significant insights, as demonstrated by Balan et al.~\cite{balanRealTimeTripInformation2011}'s study of 250 million taxi trips. Their findings indicate that urban mobility follows predictable patterns, with normal routes clustering around preferred paths between locations, and these patterns exhibit sufficient repetition to enable statistical prediction. For synthetic data generation, this emphasizes the importance of preserving origin-destination flow patterns and route clustering rather than generating entirely novel trajectory types.

Scalability represents an important practical consideration for synthetic data generation. Wu et al.~\cite{wuSafetySpatialFeature2024} demonstrate that modern anomaly detection requires distributed processing approaches to handle large datasets effectively. Consequently, synthetic data generation methods must produce datasets of sufficient scale and appropriate structure for parallel processing systems.

\subsection{Synthetic Trajectory Data Generation}
\label{sec:generation-review}

Synthetic trajectory generation has evolved rapidly from foundational map matching techniques~\cite{newsonHiddenMarkovMap2009} to sophisticated deep learning frameworks~\cite{caoGeneratingMobilityTrajectories2021,wangGTGGeneralizableTrajectory2025}. This evolution is driven by converging research pressures across multiple domains. What began as solutions to GPS noise and sparsity issues has expanded to address fundamental challenges in trajectory research.

Three critical problems drive this development. First, the parameter sensitivity and labeled data scarcity issues identified in trajectory anomaly detection research (Section~\ref{sec:anomaly-review})~\cite{zhangIBATDetectingAnomalous2011} make it difficult to systematically evaluate detection algorithms. Second, the high re-identification risk that makes real trajectory data unsuitable for research sharing~\cite{raoCATSConditionalAdversarial2023} creates fundamental data access barriers. Third, the need for reproducible evaluation frameworks that traditional privacy methods cannot provide limits research reproducibility.

This convergence shows a fundamental research gap that existing approaches struggle to address simultaneously. Traditional privacy-preserving mechanisms like k-anonymity and differential privacy create utility-privacy trade-offs that render data unsuitable for complex analytical tasks~\cite{jordonPATEGANGeneratingSynthetic2019}. Meanwhile, the controlled datasets needed for systematic anomaly detection evaluation remain unavailable.

Synthetic trajectory generation addresses these challenges by creating artificial datasets that preserve essential mobility patterns for research purposes without exposing individual trajectories~\cite{caoGeneratingMobilityTrajectories2021}. However, success requires solving complex pattern preservation problems across spatial, temporal, and behavioral dimensions~\cite{kongMobilityTrajectoryGeneration2023,merhiSyntheticTrajectoryGeneration2024}. This establishes the foundation for understanding why comprehensive privacy protection mechanisms are essential for practical deployment of synthetic trajectory generation systems.

\subsubsection{Evolution of Generation Approaches}

The development of synthetic trajectory generation shows two major research transitions that directly impact anomaly detection utility. Early foundational work and deep learning breakthroughs established the core requirements for pattern preservation, while advanced frameworks address the integration challenges essential for practical deployment.

\textbf{From Foundational Methods to Deep Learning Solutions.} Early trajectory processing research reveals fundamental insights that remain critical for anomaly detection today. Region representation learning~\cite{wangRegionRepresentationLearning2017} and map matching techniques~\cite{newsonHiddenMarkovMap2009} show how spatial relationships must be preserved to maintain the trajectory characteristics that detection algorithms depend on. These insights directly address the spatial traversal sequence requirements identified in isolation-based detection methods like iBAT (Section~\ref{sec:anomaly-review}).

The deep learning transition created a paradigm shift through GAN-based approaches like TrajGen~\cite{caoGeneratingMobilityTrajectories2021}. These approaches demonstrated that neural networks can capture complex spatio-temporal relationships while revealing fundamental challenges in temporal dependency modeling. Vehicle-specific investigations~\cite{bajarunasGenerativeAdversarialNetworks2022} highlight a key insight: GANs excel at spatial modeling but struggle with temporal sequences. This directly impacts the subtle temporal patterns and sequence dependencies that autoencoder-based detection methods require (Section~\ref{sec:anomaly-review})~\cite{huangLSTMAutoencodersAttention2021}.

\textbf{Diffusion Models for Trajectory Generation.} A significant advancement in trajectory generation comes from the adoption of diffusion probabilistic models. DiffTraj~\cite{zhuDiffTrajGeneratingGPS2023} represents a breakthrough approach that combines the generative capabilities of diffusion models with spatial-temporal features derived from real trajectories. The core innovation lies in reconstructing and synthesizing geographic trajectories from white noise through a reverse trajectory denoising process, effectively addressing the training stability issues that plagued earlier GAN-based approaches.

DiffTraj introduces a Trajectory UNet (Traj-UNet) deep neural network to embed conditional information and accurately estimate noise levels during the reverse process. This architecture demonstrates superior performance in generating high-fidelity trajectories while retaining original distributions, significantly outperforming other methods in geo-distribution evaluations. The model's ability to work directly with continuous GPS coordinates makes it particularly suitable for preserving the precise spatial patterns required for effective anomaly detection. These capabilities form the foundation for the methodology presented in Section~\ref{sec:methodology}, where DiffTraj is integrated with LM-TAD for privacy-preserving anomaly detection.

Building on this foundation, Diff-RNTraj~\cite{weiDiffRNTrajStructureawareDiffusion2024} addresses a critical limitation in practical applications by introducing road network-constrained trajectory generation. This structure-aware diffusion model generates trajectories directly on road networks with road-related information, ensuring practical utility while maintaining the generative quality of diffusion approaches. The model handles hybrid trajectory data combining discrete road segments with continuous movement rates, incorporating pre-training strategies and novel loss functions to enhance spatial validity.

These limitations drove architectural innovations including CNN-based transformations~\cite{merhiSyntheticTrajectoryGeneration2024} for spatial distribution capture and RNN approaches~\cite{duRecurrentMarkedTemporal2016} for sequential dependencies. Each approach addresses different aspects of preserving anomaly detection utility.

\textbf{Advanced Integration Approaches.} Recognition of individual approach limitations drives sophisticated hybrid methods that address comprehensive anomaly detection requirements. The Act2Loc framework~\cite{liuAct2LocSyntheticTrajectory2023} shows how machine learning can combine with mechanistic models for enhanced pattern preservation while requiring minimal training data.

Two-stage generation frameworks like TS-TrajGen~\cite{jiangContinuousTrajectoryGeneration2023} solve error accumulation problems through separated structural and continuous generation. These approaches integrate domain knowledge with model-free learning. Cross-city generalization research~\cite{wangGTGGeneralizableTrajectory2025} demonstrates scalable pattern extraction across urban environments, directly addressing the distributed processing and scalability requirements identified for modern anomaly detection systems (Section~\ref{sec:anomaly-review}).

\subsubsection{Architectural Specialization and Paradigm Shifts}

Different architectural approaches excel at capturing distinct aspects of trajectory data, directly impacting pattern preservation required for anomaly detection effectiveness.

\textbf{Sequential vs. Spatial Processing Trade-offs.} The temporal dependencies in trajectory data drive investigation of sequential architectures to address pattern requirements identified in deep learning anomaly detection approaches (Section~\ref{sec:anomaly-review}). The RMTPP framework~\cite{duRecurrentMarkedTemporal2016} shows how RNNs can model event timings and spatial markers simultaneously. This demonstrates the importance of temporal pattern preservation for sequence-dependent detection methods like LSTM-AE-Attention models.

However, RNN-based GANs exhibit training instability compared to CNN models~\cite{merhiSyntheticTrajectoryGeneration2024}. This creates trade-offs between temporal modeling capability and training reliability. Convolutional approaches like the RTCT method~\cite{merhiSyntheticTrajectoryGeneration2024} solve spatial distribution challenges through novel data transformations. Conv1D layers demonstrate superior performance for capturing spatial distributions needed for density-based anomaly detection methods that rely on local neighborhood structures (Section~\ref{sec:anomaly-review}).

This research shows a fundamental insight: CNNs excel at spatial pattern capture but struggle with sequential properties, while RNNs handle temporal dependencies but face convergence challenges.

\textbf{Language Model Paradigm Shift.} Recent advances reconceptualize trajectory generation by treating trajectories as sequences where each spatio-temporal point acts as a "word"~\cite{zhangEndtoendTrajectoryGeneration2025}. This approach addresses both sequential dependencies and spatial constraints simultaneously through autoregressive modeling. Training on finite vocabulary of locations implicitly enforces spatio-temporal validity constraints~\cite{kongMobilityTrajectoryGeneration2023}.

A prime example of this paradigm is TrajGPT~\cite{hsuTrajGPTControlledSynthetic2024}, which introduces controlled synthetic trajectory generation using a multitask transformer-based spatiotemporal model. TrajGPT addresses the novel problem of "controlled" trajectory generation, where specific gaps in partially specified sequences must be filled while maintaining spatiotemporal consistency. Unlike traditional next-location prediction methods, TrajGPT treats trajectory generation as a text infilling problem, leveraging advances in large language models to handle complex spatiotemporal relationships.

The model integrates spatial and temporal components through a Bayesian probability framework within a transformer architecture, ensuring that generated trajectories maintain coherent spatiotemporal patterns. TrajGPT demonstrates significant improvements in temporal accuracy (26-fold improvement) while maintaining over 98\% spatial accuracy, highlighting the potential of language model approaches for trajectory generation tasks.

This paradigm shift leverages broader AI advances to potentially resolve the architectural trade-offs identified in earlier approaches. Zhang et al.~\cite{zhangEndtoendTrajectoryGeneration2025} provide a comprehensive comparison between deep generative models and language models for trajectory generation, demonstrating that language model approaches can effectively capture complex trajectory patterns while maintaining computational efficiency. The approach maintains compatibility with privacy protection mechanisms and cross-city generalization through space syntax theory~\cite{wangGTGGeneralizableTrajectory2025}.

\subsubsection{Privacy-Utility Trade-offs as Design Constraints}

Privacy requirements fundamentally constrain synthetic trajectory generation approaches, creating a central tension that shapes architectural choices and evaluation frameworks. Rather than being an additional feature, privacy preservation emerges as a core design constraint that determines the feasibility and effectiveness of generation methods.

\textbf{Privacy Integration and Evaluation Challenges.} Privacy guarantees require architectural modifications that fundamentally alter generation training processes. The PATE-GAN framework~\cite{jordonPATEGANGeneratingSynthetic2019} demonstrates how differential privacy guarantees modify training to ensure bounded individual influence, while privacy-preserving integration~\cite{raoCATSConditionalAdversarial2023} constrains input representations and DP-SGD integration~\cite{merhiSyntheticTrajectoryGeneration2024} constrains optimization processes. These requirements demonstrate that privacy cannot be added post-hoc but must be integrated from the ground up. Early evaluation approaches assumed utility preservation automatically maintained research value, but privacy-specific metrics like Trajectory-User Linking~\cite{raoCATSConditionalAdversarial2023} show that utility-preserving synthetic data can still leak sensitive information, while the Synthetic Ranking Agreement metric~\cite{jordonPATEGANGeneratingSynthetic2019} demonstrates the need for careful privacy-utility balance.

\textbf{Anomaly Detection Requirements Under Privacy Constraints.} The challenge of maintaining anomaly detection research utility under privacy constraints creates specific requirements that generation methods must satisfy. The need to preserve pattern complexity for deep learning approaches while preventing high re-identification risks creates a design space where privacy constraints and research utility requirements must be jointly optimized rather than sequentially addressed. This fundamental tension determines both the feasibility of privacy-preserving synthetic generation and its effectiveness for anomaly detection research applications.

\subsubsection{Research Gaps and Synthesis Requirements}

The convergence of synthetic trajectory generation research with anomaly detection requirements shows specific gaps that current approaches struggle to address systematically. These gaps represent concrete research opportunities where advances could significantly impact both fields.

\textbf{Pattern Preservation and Evaluation Under Privacy Constraints.} Current synthetic generation methods address either pattern preservation or privacy protection effectively, but struggle with both simultaneously. While isolation-based detection methods like iBAT require the specific origin-destination flow patterns and spatial traversal sequences identified in Section~\ref{sec:anomaly-review}~\cite{zhangIBATDetectingAnomalous2011}, existing privacy-preserving approaches cannot guarantee these patterns survive the protection process.

Similarly, deep learning approaches need large, diverse datasets and subtle temporal patterns that autoencoder-based detection requires~\cite{huangLSTMAutoencodersAttention2021}. However, privacy constraints limit access to necessary training data. Existing evaluation approaches assess utility and privacy independently, but anomaly detection research requires understanding how privacy protection affects detection performance specifically. The Synthetic Ranking Agreement metric~\cite{jordonPATEGANGeneratingSynthetic2019} provides a starting point, but does not address whether synthetic data preserves the specific anomaly characteristics that detection algorithms depend on.

\textbf{Scalability and Systematic Evaluation.} The controlled nature of synthetic datasets could solve the parameter sensitivity and labeled data scarcity issues identified in anomaly detection research (Section~\ref{sec:anomaly-review})~\cite{zhangIBATDetectingAnomalous2011}, but current generation methods do not provide systematic evaluation capabilities needed. Cross-city generalization research~\cite{wangGTGGeneralizableTrajectory2025} shows promise for geographical constraints, but does not solve the fundamental challenge of generating large-scale datasets with controlled anomaly characteristics for systematic algorithm evaluation.

\textbf{Comprehensive Integration Framework.} While individual advances in generation architectures (Section~\ref{sec:generation-review}), privacy protection (Section~\ref{sec:privacy-review}), and evaluation methods show promise, no integrated framework addresses the combined requirements of anomaly detection research under privacy constraints. This creates the research opportunity for comprehensive frameworks that can handle the complexity and scale of modern urban transportation networks while maintaining both privacy protection and anomaly detection utility. Such frameworks require seamless integration of the anomaly detection requirements (Section~\ref{sec:anomaly-review}), generation capabilities, and privacy protection mechanisms examined across this literature review.

\subsection{Privacy Protection Methods}
\label{sec:privacy-review}

\subsubsection{Privacy Challenges in Trajectory Data}

Trajectory data, particularly from urban taxi operations, is highly unique and personalised~\cite{primaultLongRoadComputational2019,buchholzReconstructionAttackDifferential2022,maTrajectoryPrivacyProtection2021}. As few as four spatio-temporal points can uniquely identify 95\% of individuals~\cite{primaultLongRoadComputational2019,buchholzReconstructionAttackDifferential2022,maTrajectoryPrivacyProtection2021}. This rich information, including Points of Interest (POIs) like home or work, can reveal deeply sensitive personal details, such as religious beliefs or political preferences~\cite{primaultLongRoadComputational2019,buchholzReconstructionAttackDifferential2022}. The inherent challenge for anomaly detection research is balancing privacy protection with the need to preserve complex spatio-temporal patterns that detection algorithms require~\cite{buchholzSystematisationKnowledgeTrajectory2024,buchholzReconstructionAttackDifferential2022,primaultLongRoadComputational2019,naghizadePrivacyContextawareRelease2020}.

The central goal is enabling high-utility trajectory data release without revealing private information about individuals~\cite{buchholzSystematisationKnowledgeTrajectory2024,raoLSTMTrajGANDeepLearning2020,liuTrajGANsUsingGenerative2018,jinSurveyExperimentalStudy2023,maTrajectoryPrivacyProtection2021,naghizadePrivacyContextawareRelease2020}. This would allow researchers to develop and test anomaly detection systems without requiring direct access to sensitive real-world data~\cite{buchholzSystematisationKnowledgeTrajectory2024,raoLSTMTrajGANDeepLearning2020,liuTrajGANsUsingGenerative2018}.

\subsubsection{Traditional Privacy-Preserving Methods and Limitations}

Traditional privacy-preserving methods like k-anonymity have been shown to be vulnerable to various privacy attacks that exploit an adversary's background knowledge, proving unable to provide sufficient privacy protection for trajectory data~\cite{chenDifferentiallyPrivateTrajectory2011,buchholzReconstructionAttackDifferential2022,jinSurveyExperimentalStudy2023}. Similarly, conventional approaches such as suppression and generalization techniques struggle with the inherent complexity of trajectory data. These methods often destroy the spatio-temporal relationships that anomaly detection algorithms require. A broader issue impacting confidence in claimed privacy levels is that multiple foundational works on differentially private trajectory protection have been found to rely on erroneous proofs~\cite{buchholzSystematisationKnowledgeTrajectory2024,primaultDifferentiallyPrivateLocation2014,erroundaAnalysisDifferentialPrivacy2019}.

To achieve robust privacy for synthetic trajectory data, researchers must carefully select the Unit of Privacy (UoP)~\cite{buchholzSystematisationKnowledgeTrajectory2024,primaultLongRoadComputational2019}. Protecting individual locations (location-level privacy) in a trajectory is considered the weakest level. This approach is vulnerable to correlation and reconstruction attacks because it ignores intra-trajectory correlations~\cite{buchholzSystematisationKnowledgeTrajectory2024,buchholzReconstructionAttackDifferential2022,primaultDifferentiallyPrivateLocation2014,erroundaAnalysisDifferentialPrivacy2019}. Instance-level privacy (trajectory-level), where the entire trajectory is protected as one unit, offers a more promising balance for deep learning applications~\cite{buchholzSystematisationKnowledgeTrajectory2024}. These limitations of traditional methods have driven the development of synthetic data generation approaches as more viable alternatives.

\subsubsection{Synthetic Data Generation for Privacy Protection}

Given these challenges, synthetic trajectory data generation represents a promising alternative to directly protecting original data~\cite{buchholzSystematisationKnowledgeTrajectory2024,raoLSTMTrajGANDeepLearning2020,liuTrajGANsUsingGenerative2018}. The approach creates new, non-real trajectories that mimic the statistical and behavioral properties of the authentic data. These synthetic trajectories can then be freely shared for research and development without privacy concerns attached to specific individuals~\cite{raoLSTMTrajGANDeepLearning2020,liuTrajGANsUsingGenerative2018,quGenerativeAdversarialNetworks2020}.

Deep learning approaches, particularly Generative Adversarial Networks (GANs), have emerged as a key direction for privacy-preserving synthetic trajectory data generation~\cite{buchholzSystematisationKnowledgeTrajectory2024,liuTrajGANsUsingGenerative2018,raoLSTMTrajGANDeepLearning2020,quGenerativeAdversarialNetworks2020}. Liu et al.~\cite{liuTrajGANsUsingGenerative2018} proposed trajGANs to generate synthetic trajectories that preserve the summary properties of real data and achieve close-to-real-data performance in analysis tasks. These privacy-focused approaches build on the generation capabilities and architectural trade-offs discussed in Section~\ref{sec:generation-review} to specifically address data protection requirements~\cite{raoLSTMTrajGANDeepLearning2020,quGenerativeAdversarialNetworks2020,buchholzSystematisationKnowledgeTrajectory2024,ponomarevaHowDPfyML2023}.

For urban taxi operations, generative models have been evaluated on real-world datasets like the T-Drive dataset (Beijing taxi trajectories) and the San Francisco cabs dataset~\cite{maTrajectoryPrivacyProtection2021,primaultDifferentiallyPrivateLocation2014,primaultLongRoadComputational2019}. These datasets capture the specific spatio-temporal continuity and regularity of taxi movements, which models like LSTM-TrajGAN are designed to preserve~\cite{raoLSTMTrajGANDeepLearning2020,liuTrajGANsUsingGenerative2018,jinSurveyExperimentalStudy2023}. Preserving spatial and temporal characteristics is crucial for anomaly detection, as anomalies are deviations from learned normal patterns. This requirement creates particular challenges for privacy-preserving methods, as the complex patterns needed for effective anomaly detection (Section~\ref{sec:anomaly-review}) must be maintained while protecting individual privacy~\cite{raoLSTMTrajGANDeepLearning2020,naghizadePrivacyContextawareRelease2020}.

Alternative approaches include DP-driven synthetic methods such as DPT (Differentially Private Trajectory Synthesis), which adapts the Laplacian mechanism and uses hierarchical reference systems to model trajectories~\cite{chenDifferentiallyPrivateTrajectory2011,jinSurveyExperimentalStudy2023}. AdaTrace builds upon DPT by incorporating attack resilience and a utility-aware generator, generally outperforming DPT in utility preservation~\cite{jinSurveyExperimentalStudy2023}. These models aim to capture the statistical distribution of the original data to sample synthetic trajectories while providing formal privacy guarantees~\cite{jinSurveyExperimentalStudy2023,quGenerativeAdversarialNetworks2020}.

\subsubsection{Privacy Evaluation and Open Challenges}

The ultimate aim of generating synthetic trajectory datasets is to replace original trajectories for data sharing and publication~\cite{buchholzSystematisationKnowledgeTrajectory2024,raoLSTMTrajGANDeepLearning2020,liuTrajGANsUsingGenerative2018}. This directly addresses the need for anomaly detection research in urban taxi operations to proceed without requiring access to sensitive real-world data. Such an approach would overcome privacy concerns and regulatory hurdles associated with using actual mobility traces~\cite{buchholzSystematisationKnowledgeTrajectory2024,raoLSTMTrajGANDeepLearning2020,liuTrajGANsUsingGenerative2018}. Synthetic data must support diverse analytical tasks including spatial and temporal analyses, classification, clustering, and anomaly detection while maintaining utility for research purposes~\cite{raoLSTMTrajGANDeepLearning2020,chenDifferentiallyPrivateTrajectory2011}. The CC-Net system demonstrates privacy-preserved taxi demand prediction, achieving high accuracy while ensuring privacy by design~\cite{ozekiBalancingPrivacyUtility2023}.

Despite progress, significant challenges remain unresolved. No existing solution satisfies all requirements for fully private and high-utility synthetic trajectory data~\cite{buchholzSystematisationKnowledgeTrajectory2024}. The lack of standardisation in evaluation metrics and frameworks continues to make direct comparisons challenging~\cite{primaultLongRoadComputational2019,jinSurveyExperimentalStudy2023}. The assessment of privacy guarantees requires diligent verification for any proposed synthetic data generation method~\cite{buchholzSystematisationKnowledgeTrajectory2024}. Future research must focus on developing novel privacy-preserving trajectory publication mechanisms that provide both high levels of utility and privacy, and are not susceptible to reconstruction attacks~\cite{buchholzReconstructionAttackDifferential2022,buchholzSystematisationKnowledgeTrajectory2024,primaultLongRoadComputational2019}. The design of a fully differentially private generative model for trajectories that captures complex spatio-temporal patterns while resisting sophisticated attacks remains a compelling and urgent open research question~\cite{buchholzSystematisationKnowledgeTrajectory2024,buchholzReconstructionAttackDifferential2022}.

\subsection{Synthesis and Research Framework}
\label{sec:synthesis}

The comprehensive examination of trajectory anomaly detection, synthetic data generation, and privacy protection reveals a critical convergence point that defines the research opportunity addressed in this thesis. The three research areas exhibit complementary strengths and limitations that, when properly integrated, create a pathway to address fundamental challenges in privacy-preserving trajectory research.

\subsubsection{Convergence of Research Requirements}

The analysis demonstrates that trajectory anomaly detection, synthetic data generation, and privacy protection share fundamental requirements that must be addressed simultaneously rather than sequentially. Anomaly detection algorithms require specific pattern preservation capabilities: origin-destination flow patterns and spatial traversal sequences for isolation-based methods (Section~\ref{sec:anomaly-review}), subtle temporal patterns and sequence dependencies for deep learning approaches, and time-dependent behavioral patterns for comprehensive spatio-temporal analysis.

Synthetic data generation approaches have developed sophisticated capabilities to address these requirements through architectural innovations including CNN-based spatial modeling, RNN-based temporal processing, and language model paradigms that handle both spatial and temporal constraints (Section~\ref{sec:generation-review}). However, these generation capabilities face fundamental constraints when privacy protection mechanisms are integrated, as differential privacy guarantees, trajectory-level protection, and attack resistance requirements fundamentally alter training processes and pattern preservation capabilities.

Privacy protection research identifies the critical challenge that traditional privacy-preserving methods destroy the spatio-temporal relationships essential for accurate anomaly detection (Section~\ref{sec:privacy-review}). This creates a design space where privacy constraints and research utility requirements must be jointly optimized. The 95\% individual identification risk from just four spatio-temporal points demonstrates why privacy cannot be treated as a post-processing step, but must be integrated throughout the entire research pipeline.

\subsubsection{Integrated Framework Requirements}

The convergence analysis reveals that effective privacy-preserving trajectory anomaly detection requires an integrated framework that addresses five core challenges simultaneously:

\textbf{Pattern Preservation Under Privacy Constraints.} The framework must preserve the specific trajectory characteristics that anomaly detection algorithms require while providing strong privacy guarantees. This requires understanding how privacy protection mechanisms affect the spatial traversal sequences, temporal dependencies, and behavioral patterns that different detection methods depend on.

\textbf{Scalable Synthetic Generation.} The framework must generate synthetic datasets of sufficient scale and diversity to support systematic evaluation of anomaly detection algorithms while maintaining computational efficiency for practical deployment. This addresses the parameter sensitivity and labeled data scarcity issues identified in anomaly detection research.

\textbf{Comprehensive Privacy Protection.} The framework must provide robust privacy protection against sophisticated attacks while preserving utility for anomaly detection research. This requires careful selection of privacy units, integration of multiple protection mechanisms, and evaluation against both membership inference and reconstruction attacks.

\textbf{Systematic Evaluation Capabilities.} The framework must enable systematic evaluation of anomaly detection methods through controlled synthetic datasets with known anomaly characteristics. This addresses the reproducibility and comparison challenges that limit current anomaly detection research.

\textbf{Practical Deployment Considerations.} The framework must address the computational requirements, scalability constraints, and cross-city generalization needs for practical deployment in urban transportation systems.

\subsubsection{Research Contribution and Methodology Framework}

This thesis addresses the identified convergence point by developing a comprehensive framework that integrates LM-TAD-based anomaly detection, statistical pattern extraction, and privacy-preserving synthetic generation. The approach leverages the strengths identified in each research area while addressing their individual limitations through systematic integration.

The methodology framework builds on LM-TAD analysis to understand both normal and anomalous trajectory patterns in real data, extracting the specific statistical and behavioral properties that must be preserved in synthetic generation. The framework implements multiple privacy protection mechanisms designed to work together rather than independently, ensuring that privacy guarantees do not compromise the pattern preservation essential for anomaly detection research.

The synthetic generation component addresses the architectural trade-offs identified in existing approaches by combining spatial modeling capabilities with temporal pattern preservation, while the privacy protection mechanisms ensure that the resulting synthetic data provides strong privacy guarantees without destroying research utility. The comprehensive evaluation framework enables systematic assessment of both privacy protection and anomaly detection performance, addressing the evaluation gaps identified across all three research areas.

This integrated approach creates a research contribution that extends beyond individual advances in any single area, providing a complete solution for privacy-preserving trajectory anomaly detection research that addresses the fundamental challenges identified in each research domain while enabling practical deployment in urban transportation systems.

\subsection{Literature-Informed Methodology Framework}
\label{sec:lit-methodology-bridge}

The literature review analysis directly informs the methodological choices presented in this thesis. The identified convergence of DiffTraj's generation capabilities~\cite{zhuDiffTrajGeneratingGPS2023}, LM-TAD's anomaly detection paradigm~\cite{mbuyaTrajectoryAnomalyDetection2024}, and the controlled generation insights from TrajGPT~\cite{hsuTrajGPTControlledSynthetic2024} creates a unique opportunity for integration. The methodology leverages DiffTraj's demonstrated training stability and high-fidelity generation, LM-TAD's interpretable perplexity-based anomaly scoring, and incorporates the iterative refinement principles that address the pattern preservation requirements identified across the literature.

This literature-informed approach directly addresses the research gaps identified in Section~\ref{sec:synthesis}, particularly the need for systematic evaluation capabilities under privacy constraints and the challenge of maintaining anomaly detection utility while providing strong privacy guarantees.


\begin{abstract}
  Urban traffic management, transportation planning, and intelligent city systems require accurate real-time trajectory prediction to support policy decisions and optimize traffic flow. However, existing fast prediction models suffer from poor route completion rates (12-18\%), limiting their practical deployment for traffic regulators and urban planners. While sophisticated models like LM-TAD achieve superior spatial reasoning, their computational overhead (~3.4ms per trajectory vs ~0.1ms for fast models) prevents real-time application in city-wide traffic management systems, digital twin platforms, and large-scale simulations.

  This thesis addresses this challenge through training-time knowledge distillation, transferring spatial understanding from LM-TAD (a trajectory anomaly detection model) to HOSER (a fast zone-based prediction model). We demonstrate that repurposing the ``normal trajectory'' knowledge learned by anomaly detection models enables dramatic improvements in route prediction without inference-time overhead. Our distillation framework achieves 85-89\% path completion success (47-74$\times$ improvement over vanilla baseline), 87\% better distance distribution matching, and 98\% better spatial pattern fidelity on Beijing's 40,060-road network with 629,380 training trajectories. Hyperparameter optimization reveals that minimal distillation weight ($\lambda$=0.0014) with high temperature ($\tau$=4.37) enables effective knowledge transfer while preserving the student model's fast inference speed.

  The resulting system enables practical deployment for policy makers and traffic regulators, supporting applications in real-time traffic signal optimization, infrastructure planning, urban digital twins, agent-based traffic simulation, and high-quality synthetic trajectory data generation for training other models. This work demonstrates the viability of cross-task knowledge distillation for trajectory prediction and provides a scalable framework for integrating AI-based route prediction into operational traffic management systems.

  \keywords{Knowledge distillation \and Trajectory prediction \and Urban transportation \and Traffic management \and Digital twins \and Deep learning}
\end{abstract}


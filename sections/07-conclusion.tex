% Source: Synthesized from abstract, EVALUATION_ANALYSIS.md Section 8, and writing notes

\section{Conclusion}
\label{sec:conclusion}

This thesis addresses a fundamental challenge in urban trajectory prediction: how to achieve transformer-level spatial reasoning while maintaining millisecond-scale inference speeds required for real-time traffic management. Through training-time knowledge distillation, we transfer spatial understanding from LM-TAD (a trajectory anomaly detection model) to HOSER (a fast zone-based prediction model), demonstrating that cross-task knowledge transfer can dramatically improve route prediction without inference-time overhead.

\subsection{Summary of Contributions}
\label{sec:conclusion-contributions}

This thesis makes four primary contributions to trajectory prediction research (detailed in Section~\ref{sec:introduction}):

\textbf{1. Cross-Task Distillation for Trajectory Generation.} We propose the first knowledge distillation framework transferring spatial reasoning from trajectory anomaly detection (LM-TAD) to trajectory prediction (HOSER). On Beijing, this enables dramatic generation improvements: 85--89\% path completion versus vanilla's 12\%, 87--98\% JSD reductions, with no inference-time overhead. This cross-task paradigm demonstrates that ``normal trajectory'' knowledge learned by anomaly detectors provides valuable spatial priors for route prediction.

\textbf{2. Scenario-Level Evaluation Protocol and Context-Dependent Findings.} We introduce OD matching as the primary end-to-end metric and develop a scenario-level evaluation protocol (9 spatial/temporal contexts). This reveals context-dependent distillation behavior: universal benefits on complex urban networks (Beijing: all scenarios improve) versus spatially localized improvements in simpler environments (Porto: dense urban gains, suburban neutral).

\textbf{3. Validation--Generation Disconnect as Methodological Insight.} Hyperparameter optimization shows minimal validation accuracy gains (+0.01--0.23\%) produce dramatic generation quality improvements (+73\% OD match on Beijing). This disconnect demonstrates that next-step prediction accuracy is a poor proxy for long-horizon trajectory realism, motivating multi-objective optimization targeting trajectory quality metrics (JSD, OD match, Hausdorff) rather than validation accuracy surrogates.

\textbf{4. Dataset-Specific Hyperparameter Necessity.} Systematic Optuna-based optimization reveals non-transferable optimal configurations across cities: $\lambda$ diverges by 4.3$\times$, $\tau$ by 42\%, $w$ by 43\% between Beijing and Porto (Table~\ref{tab:distill-hparams}). This demonstrates per-dataset HPO is mandatory. The batch-size confound (Porto requires 4$\times$ reduction due to longer trajectories) complicates interpretation but highlights critical experimental design considerations for future work.

\subsection{Key Findings}
\label{sec:conclusion-findings}

Our experiments reveal several important insights about knowledge distillation for trajectory prediction:

\textbf{Knowledge Distillation Transfers Spatial Understanding.} The dramatic improvements in path completion (85--89\% vs 12--18\%) and distribution quality (87--98\% JSD reduction) demonstrate that distillation transfers \emph{fundamental spatial reasoning}, not merely improved metrics. Vanilla HOSER systematically generates unrealistically short trips (2.4 km) and fails to reach destinations, while distilled models navigate successfully and produce realistic-length routes.

\textbf{Minimal Guidance with Broad Knowledge Works Best.} The optimal configuration uses very low distillation weight ($\lambda = 0.0014$) but high temperature ($\tau = 4.37$). This result is counter-intuitive: conventional knowledge distillation typically employs $\lambda$ values between 0.1 and 0.9, meaning the teacher loss contributes 10--90\% to the training objective. Our optimal $\lambda = 0.0014$ means distillation accounts for only \emph{0.14\% of the total loss}—the student learns 99.86\% from ground truth and barely 0.14\% from the teacher. At first glance, this raises the question: why bother with distillation at all? The answer lies in \emph{what kind} of knowledge is transferred: high temperature ($\tau = 4.37$) produces soft, broadly distributed targets that encode relational information across many candidate roads. Even with minimal weight, this distributional signal provides valuable spatial priors. Recent work validates this phenomenon~\cite{pengAdaSwitchAdaptiveSwitching2025,singhORPODistillMixedPolicyPreference2025}: minimal teacher intervention preserves student capacity while preventing the distribution narrowing that occurs with excessive supervision. The finding suggests that \emph{quality of guidance matters more than quantity}—subtle, broad knowledge transfer outperforms heavy-handed supervision.

\textbf{Distillation Enables True Generalization.} Distilled models perform \emph{better on test OD pairs than training OD pairs} (lower JSD), indicating they learned generalizable spatial patterns rather than memorizing training routes. This counter-intuitive result suggests the teacher's distributional knowledge helps students abstract beyond specific trajectory examples.

\textbf{Vanilla HOSER Has Fundamental Spatial Limitations.} Without distillation, HOSER suffers from severe spatial reasoning deficits: (i) 82--88\% path completion failure, (ii) 55\% underestimation of trip lengths, and (iii) 50--70$\times$ worse spatial complexity modeling. These are not merely quantitative differences but fundamental failures that prevent practical deployment.

\textbf{Knowledge Transfer Is Robust.} Cross-seed evaluation (seeds 42, 44) shows coefficient of variation below 15\% for all metrics, confirming distillation reliably transfers knowledge regardless of initialization. The consistency across random seeds validates the framework's stability for production use.

\subsection{Practical Impact}
\label{sec:conclusion-impact}

The resulting system enables several practical applications for urban traffic management and intelligent transportation:

\textbf{Real-Time Traffic Management.} Fast inference speeds (measured 1.9 trajectories/second on commodity hardware, Table~\ref{tab:generation-perf}) combined with accurate route prediction support real-time traffic signal optimization, dynamic routing, and congestion management at city scale.

\textbf{Infrastructure Planning and Policy Decisions.} High-quality trajectory generation enables traffic regulators to simulate infrastructure changes (new roads, lane additions, traffic calming) and predict their impact on mobility patterns before costly construction.

\textbf{Urban Digital Twins.} Realistic trajectory synthesis supports digital twin platforms that mirror real city dynamics, enabling what-if analysis for urban planning, emergency response simulation, and long-term development strategies.

\textbf{Agent-Based Traffic Simulation.} Generated trajectories can populate large-scale agent-based simulations with diverse, realistic mobility patterns, supporting research in autonomous vehicles, shared mobility, and transportation network optimization.

\textbf{Model Evaluation and Testing.} The framework generates high-fidelity synthetic trajectories that can be used to evaluate and test other trajectory-based models (e.g., travel time estimators, destination predictors, routing systems) with realistic mobility patterns.

\subsection{Limitations}
\label{sec:conclusion-limitations}

Despite promising results, several limitations warrant acknowledgment:

\textbf{Limited Dataset Evaluation.} We present complete results for Beijing and Porto (Phase 1 hyperparameters). Comprehensive cross-dataset validation with additional urban environments is needed to confirm generalization across cities with different characteristics (network topology, trajectory lengths, mobility patterns).

\textbf{Incomplete Inference Speed Benchmarking.} While we have measured Porto generation throughput (Table~\ref{tab:generation-perf}), Beijing measurements were not collected, and formal teacher-vs-student latency comparisons remain future work. Complete characterization across datasets, batch sizes, and hardware configurations is needed for production deployment guidance.

\textbf{Limited Ablation Studies.} We lack comprehensive ablation studies for key design choices:
\begin{itemize}[noitemsep,topsep=0pt]
    \item Distillation weight sensitivity ($\lambda$ from 0 to 1)
    \item Learning rate influence (noted as impactful but not systematically studied)
    \item Temperature sensitivity beyond Optuna's explored range
    \item Alternative teacher models or multi-teacher configurations
\end{itemize}

\textbf{Single Teacher-Student Pair.} We evaluate only LM-TAD $\rightarrow$ HOSER distillation. Whether the benefits generalize to other teacher-student combinations (e.g., other anomaly detectors, different prediction models) remains unknown.

\textbf{Architectural Constraints.} HOSER's hierarchical zone-based architecture may limit applicability to other trajectory prediction models with different architectural paradigms (pure transformers, diffusion models, etc.). The framework requires vocabulary alignment mechanisms specific to each architecture.

\textbf{Evaluation Limitations.} The OD pair matching metric (grid-based, 111m resolution) may be sensitive to grid size choice. Alternative evaluation protocols (e.g., corridor-based matching, semantic location matching) could provide complementary perspectives on path completion quality.

\textbf{Validation Accuracy as Poor Proxy for Generation Quality.} Hyperparameter optimization based on validation next-step accuracy (+0.01--0.23\% improvement) produces dramatic generation quality differences (+73\% OD match improvement on Beijing). This disconnect indicates that validation metrics measure local decision accuracy but fail to capture long-horizon spatial understanding required for realistic end-to-end trajectories. Future work should explore multi-objective optimization directly targeting trajectory quality metrics (JSD, OD match, Hausdorff distance) rather than relying on validation accuracy as a surrogate.

\subsection{Future Work}
\label{sec:conclusion-future}

Several promising directions extend this research:

\subsubsection{Extended Dataset Evaluation}

\textbf{Complete Porto Phase 2 Evaluation.} Finish evaluation with Phase 2 optimal hyperparameters ($\lambda = 0.00598$, $\tau = 2.515$, $w = 4$) to compare against Phase 1 results and validate final performance.

\textbf{BJUT Abnormal-Trajectory Evaluation (Work in Progress).} Planned targeted case study using the Beijing Private (BJUT) taxi dataset for cross-dataset validation of distillation effectiveness on an independent data source. Given current data quality limitations (insufficient clean sample size for large-scale evaluation), this analysis is underway and targeted for inclusion in the final thesis version. The evaluation will assess whether distillation benefits generalize beyond HOSER's curated benchmark datasets.

\textbf{Additional Urban Networks.} Evaluate on diverse cities (Chengdu, Xi'an, San Francisco, London) covering varied network topologies (grid vs organic street patterns), scales (dense metropolitan vs sprawling suburban), and mobility patterns (taxi-dominated vs mixed-mode transportation).

\textbf{Cross-Dataset Transfer.} Investigate whether a teacher trained on Beijing can distill effectively for Porto students, enabling knowledge transfer across cities without retraining teachers for each location.

\subsubsection{Systematic Ablation Studies}

\textbf{Distillation Weight Sensitivity.} Conduct $\lambda$ ablation from 0 to 1 to understand the full influence curve. Particular focus on: (i) why minimal $\lambda = 0.0014$ is optimal, (ii) whether $\lambda = 1.0$ (pure distillation) completely fails, and (iii) the shape of the performance vs $\lambda$ relationship.

\textbf{Learning Rate Analysis.} Systematically evaluate learning rates from $10^{-5}$ to $10^{-3}$ with fixed distillation parameters. Hypothesis: lower learning rates may enable finer-grained teacher knowledge integration.

\textbf{Temperature Characterization.} Evaluate $\tau \in [1, 10]$ to map the temperature-performance relationship. Expected: very low $\tau$ provides minimal smoothing (limited dark knowledge), very high $\tau$ over-smooths and loses discriminative information.

\textbf{Batch Size and Memory Trade-offs.} Porto's longer trajectories necessitated batch size reduction (128 $\to$ 32) due to O(T$^2$) attention memory scaling with gradient checkpointing. This creates an uncontrolled confound---impossible to isolate whether cross-dataset hyperparameter divergence (4.3$\times$ higher $\lambda$, 42\% lower $\tau$, 43\% shorter $w$) stems from dataset characteristics or batch size effects. Future work should investigate: (1) controlled ablation with matched batch sizes (requires more GPU memory or trajectory subsampling), (2) efficient attention mechanisms (sparse attention, chunked processing) to enable larger batch sizes on long sequences, (3) empirical characterization of batch size impact on optimal distillation hyperparameters.

\subsubsection{Inference Speed Validation}

\textbf{Formal Benchmarking.} Conduct systematic latency measurements comparing:
\begin{itemize}[noitemsep,topsep=0pt]
    \item Vanilla vs distilled HOSER (should be identical—validate this claim)
    \item HOSER vs LM-TAD teacher (expected $\sim$33$\times$ speedup)
    \item Batch size sensitivity and optimal batch configuration
    \item Hardware-specific performance (different GPUs, CPU-only inference)
\end{itemize}

\textbf{Production Deployment Profiling.} Characterize end-to-end latency including data loading, candidate generation, model inference, and post-processing. Identify bottlenecks and optimization opportunities for real-time deployment.

\subsubsection{Extended Distillation Framework}

\textbf{Alternative Teacher Models.} Explore distillation from other spatial knowledge sources:
\begin{itemize}[noitemsep,topsep=0pt]
    \item Large trajectory foundation models~\cite{maLearningUniversalHuman2025}
    \item Graph neural networks with rich spatial embeddings
    \item Diffusion-based trajectory generators~\cite{chuSimulatingHumanMobility2024}
    \item Ensemble teachers combining multiple models
\end{itemize}

\textbf{Multi-Teacher Distillation.} Investigate whether combining knowledge from multiple teachers (e.g., anomaly detector + foundation model) provides complementary benefits. Develop strategies for weighting and integrating diverse teacher signals.

\textbf{Task-Specific Distillation.} Explore whether distillation can transfer other capabilities beyond spatial reasoning: temporal patterns, route diversity, multi-modal behavior, destination prediction.

\textbf{Progressive Distillation.} Investigate staged knowledge transfer: first distill basic spatial understanding, then refine with trajectory-specific knowledge, potentially improving convergence and final performance.

\textbf{On-Policy and Mixed-Policy Distillation.} Our framework employs off-policy distillation with frozen teacher supervision on ground-truth trajectories. Recent advances explore on-policy approaches where students generate trajectories during training~\cite{singhORPODistillMixedPolicyPreference2025,pengAdaSwitchAdaptiveSwitching2025}, potentially reducing training-inference mismatch. Industry implementations report substantial compute efficiency gains with on-policy distillation while using the same reverse KL objective~\cite{OnPolicyDistillation}. Future work could investigate: (i) on-policy trajectory generation during training; (ii) mixed-policy strategies combining fixed teacher supervision with student-generated samples; (iii) adaptive mechanisms responding to student confidence or trajectory difficulty.

\textbf{Dual-Space Knowledge Distillation.} The vocabulary mapping $\psi$ enables cross-vocabulary transfer, but distillation occurs only in the student (road-segment) space. Recent work on cross-vocabulary KD~\cite{zhangDualSpaceFrameworkGeneral2025} demonstrates that conducting distillation in both representation spaces—projecting teacher hidden states to student space AND student hidden states to teacher space—can improve knowledge transfer. Future work could explore: (i) bidirectional distillation in both road and grid spaces; (ii) trainable projectors for vocabulary alignment; (iii) quantifying the impact of vocabulary mismatch on transfer effectiveness.

\subsubsection{Theoretical Understanding}

\textbf{Why Does Minimal $\lambda$ Work Best?} Develop theoretical framework explaining why subtle teacher guidance ($\lambda = 0.0014$) outperforms stronger knowledge transfer. Connection to regularization, implicit bias, and student capacity constraints.

\textbf{Temperature-Knowledge Relationship.} Formalize the relationship between temperature, dark knowledge extraction, and student learning dynamics. When does high temperature help vs harm knowledge transfer?

\textbf{Cross-Task Transfer Analysis.} Characterize what makes anomaly detection knowledge useful for prediction. Can we predict \emph{a priori} which task combinations will yield successful distillation?

\subsubsection{Application Extensions}

\textbf{Real-Time System Integration.} Deploy distilled models in operational traffic management systems, evaluate performance under production constraints, and gather feedback from traffic regulators on practical utility.

\textbf{Federated Distillation.} Explore privacy-preserving distillation where teachers are trained on sensitive data (real trajectories) but students learn only distributional knowledge, enabling deployment without raw data exposure.

\textbf{Online Adaptation.} Investigate whether distilled models can adapt to changing traffic patterns (construction, events, seasonal variations) through online learning while maintaining spatial consistency from teacher knowledge.

\textbf{Multi-Modal Trajectory Synthesis.} Extend to other transportation modes (walking, cycling, public transit) and multi-modal journeys, enabling comprehensive urban mobility modeling.

\subsection{Concluding Remarks}
\label{sec:conclusion-remarks}

This thesis demonstrates that training-time knowledge distillation enables lightweight trajectory prediction models to achieve transformer-level spatial reasoning without inference-time computational overhead. The dramatic improvements in path completion success (47--74$\times$), distribution quality (87--98\% JSD reduction), and spatial pattern fidelity validate cross-task knowledge transfer as a powerful paradigm for trajectory prediction.

The finding that \emph{minimal distillation weight with high temperature} works best challenges conventional distillation wisdom and suggests fundamental insights about how students integrate teacher knowledge. The distilled models' ability to \emph{generalize better than they memorize} further demonstrates that distributional guidance helps students abstract beyond specific training examples.

These results have immediate practical implications for urban traffic management, enabling policy makers and traffic regulators to deploy AI-based route prediction systems that balance accuracy and efficiency. The framework supports critical applications including real-time traffic signal optimization, infrastructure planning, urban digital twins, and high-quality synthetic data generation—all requiring fast, accurate trajectory prediction at metropolitan scale.

Looking forward, the cross-task distillation paradigm opens new research directions in trajectory modeling. By combining the strengths of different model families (transformers for spatial reasoning, lightweight models for speed), distillation enables practical deployment of sophisticated AI systems in real-world urban transportation. As cities worldwide invest in intelligent transportation infrastructure and digital twin platforms, techniques like knowledge distillation will prove essential for bridging the gap between research-quality models and production-ready systems.

The journey from transformer-based anomaly detection to fast, distilled route prediction illustrates a broader principle: \emph{architectural diversity is a resource, not an obstacle}. Different models excel at different aspects of trajectory modeling. Knowledge distillation allows us to combine these strengths, creating systems that are greater than the sum of their parts. This synthesis—bringing together spatial understanding, computational efficiency, and cross-task transfer—represents a promising path toward truly intelligent urban transportation systems.


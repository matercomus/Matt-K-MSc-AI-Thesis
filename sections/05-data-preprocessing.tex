% Source: Complete rewrite for HOSER distillation project
% References: notes/hoser/docs/DATASET_SETUP.md, notes/hoser/hoser-distill-optuna-6/EVALUATION_ANALYSIS.md

\section{Datasets and Preprocessing}
\label{sec:data-preprocessing}

This section describes the datasets used for distillation experiments, their preprocessing pipeline, and the compatibility layer required to bridge HOSER's road-based and LM-TAD's grid-based representations. We evaluate our framework on three urban trajectory datasets representing different cities and data sources.

\subsection{Dataset Overview}
\label{sec:data-overview}

We evaluate the distillation framework on three urban trajectory datasets (statistics in Table~\ref{tab:dataset-stats}). The Beijing HOSER reference dataset~\cite{caoHolisticSemanticRepresentation2025} provides our primary evaluation benchmark, while Porto and BJUT datasets enable cross-dataset validation.

\textbf{Beijing HOSER Reference Dataset:} Standardized benchmark from the original HOSER paper with 40,060 road segments and 629,380 training trajectories. Map-matched by the original authors, ensuring high alignment quality. We adapted the format for LM-TAD compatibility while preserving train/validation/test splits.

\textbf{Porto HOSER Dataset:} Different urban environment with longer average trajectories (8.0 vs 4.6 road segments). Also from the original HOSER paper~\cite{caoHolisticSemanticRepresentation2025}. \textcolor{red}{[EVALUATION IN PROGRESS]}

\textbf{Beijing Private (BJUT) Dataset:} Independent data source from Beijing University of Technology for cross-dataset validation. \textcolor{red}{[TO BE COMPLETED]}

\subsection{Dataset Format}
\label{sec:data-format}

We use the HOSER dataset format~\cite{caoHolisticSemanticRepresentation2025}, which represents trajectories as sequences of map-matched road segment IDs with corresponding timestamps and explicit origin-destination pairs. The road network topology and attributes enable candidate generation for spatial pruning during prediction. Map-matching was performed by the original HOSER authors, ensuring high alignment quality.

\subsection{Vocabulary Alignment}
\label{sec:data-lmtad-compat}

LM-TAD tokenizes trajectories using a uniform spatial grid~\cite{mbuyaTrajectoryAnomalyDetection2024}, while HOSER predicts over road segments. We construct the mapping $\phi: \mathcal{V} \rightarrow \mathcal{Z}$ (Algorithm~\ref{alg:vocab-map} in \autoref{sec:method-vocab}) that assigns each road to its corresponding grid cell based on centroid coordinates, enabling teacher knowledge to inform student predictions despite the vocabulary mismatch. Grid configuration: Beijing uses $205 \times 252 = 51{,}660$ cells; Porto uses $46 \times 134 = 6{,}164$ cells.

\subsubsection{Vocabulary Alignment Statistics}

Table~\ref{tab:vocab-alignment} summarizes the vocabulary alignment characteristics for each dataset.

\begin{table}[h]
\centering
\caption{Vocabulary alignment between HOSER roads and LM-TAD grid cells}
\label{tab:vocab-alignment}
\small
\begin{tabular}{lcccc}
\toprule
\textbf{Dataset} & \textbf{Roads ($|\mathcal{V}|$)} & \textbf{Grid Size} & \textbf{Cells ($|\mathcal{Z}|$)} & \textbf{Avg. Roads/Cell} \\
\midrule
Beijing & 40,060 & $205 \times 252$ & 51,660 & 0.78 \\
Porto & 11,024 & $46 \times 134$ & 6,164 & \textcolor{red}{[TBC]} \\
BJUT & \textcolor{red}{[TBC]} & \textcolor{red}{[TBC]} & \textcolor{red}{[TBC]} & \textcolor{red}{[TBC]} \\
\bottomrule
\end{tabular}
\end{table}

This alignment enables the teacher's distributional knowledge over grid cells to inform the student's predictions over road segments, despite the vocabulary size mismatch.

\subsection{Preprocessing Pipeline}
\label{sec:data-pipeline}

Dataset preparation involves: (1) zone partitioning following HOSER's hierarchical structure~\cite{caoHolisticSemanticRepresentation2025}, (2) zone transition matrix computation from training trajectories, (3) LM-TAD teacher checkpoint preparation~\cite{mbuyaTrajectoryAnomalyDetection2024}, and (4) vocabulary mapping construction. These steps adapt the base models for distillation training. Preprocessing is efficient: zone partitioning takes $\sim$15--20 seconds for Beijing (40,060 roads), and transition matrix construction takes $\sim$10--15 seconds (629,380 trajectories).

\subsubsection{Data Splitting and OD Stratification}

HOSER's original data splits are preserved to maintain comparability:

\begin{itemize}[noitemsep,topsep=0pt]
\item \textbf{Training set}: Used for distillation and model parameter updates
\item \textbf{Validation set}: Used for hyperparameter tuning (Optuna objective)
\item \textbf{Test set}: Held out for final evaluation (never seen during training or tuning)
\end{itemize}

Critically, these splits are stratified by origin-destination (OD) pairs. The test set contains OD pairs \emph{not present in training}, enabling evaluation of generalization to unseen routes rather than mere memorization.

\subsection{Dataset Statistics}
\label{sec:data-stats}

Table~\ref{tab:dataset-stats} summarizes key statistics for each evaluation dataset.

\begin{table}[h]
\centering
\caption{Trajectory dataset statistics}
\label{tab:dataset-stats}
\small
\begin{tabular}{lccc}
\toprule
\textbf{Statistic} & \textbf{Beijing} & \textbf{Porto} & \textbf{BJUT} \\
\midrule
\multicolumn{4}{l}{\textit{Road Network}} \\
\quad Road segments & 40,060 & $\sim$11,024 & \textcolor{red}{[TBC]} \\
\quad Spatial zones & 300 & 300 & \textcolor{red}{[TBC]} \\
\quad Grid cells (LM-TAD) & 51,660 & 6,164 & \textcolor{red}{[TBC]} \\
\midrule
\multicolumn{4}{l}{\textit{Trajectories}} \\
\quad Training & 629,380 & $\sim$481,359 & \textcolor{red}{[TBC]} \\
\quad Validation & 78,673 & \textcolor{red}{[TBC]} & \textcolor{red}{[TBC]} \\
\quad Test & 179,823 & \textcolor{red}{[TBC]} & \textcolor{red}{[TBC]} \\
\midrule
\multicolumn{4}{l}{\textit{Trajectory Characteristics}} \\
\quad Avg. length (roads) & 4.6 & $\sim$8.0 & \textcolor{red}{[TBC]} \\
\quad Avg. distance (km) & 5.16 & \textcolor{red}{[TBC]} & \textcolor{red}{[TBC]} \\
\quad Avg. duration (min) & 28.2 & \textcolor{red}{[TBC]} & \textcolor{red}{[TBC]} \\
\midrule
\multicolumn{4}{l}{\textit{Preprocessing}} \\
\quad Map-matching quality & High (HOSER authors) & High (HOSER authors) & \textcolor{red}{[Independent]} \\
\quad Partition time (sec) & 15--20 & 4 & \textcolor{red}{[TBC]} \\
\quad Zone trans. matrix (sec) & 10--15 & 67 & \textcolor{red}{[TBC]} \\
\bottomrule
\end{tabular}
\end{table}

\textbf{Note on trajectory length:} Porto trajectories are substantially longer than Beijing (8.0 vs 4.6 road segments on average), leading to quadratic memory scaling in attention mechanisms. This necessitates reduced batch sizes for Porto experiments (see \autoref{sec:impl-practical}).

\subsection{Data Quality and Representativeness}
\label{sec:data-quality}

The datasets represent real-world urban mobility patterns with inherent characteristics:

\textbf{Beijing:} High-density metropolitan area with complex road hierarchy. Trajectories span residential, commercial, and transportation hub regions, providing diverse route types.

\textbf{Porto:} Smaller city with different urban structure. Longer trajectories suggest different mobility patterns (potentially more suburban routes or lower road density).

\textbf{BJUT (planned):} Independent data source enables assessment of whether distillation benefits generalize beyond HOSER's curated benchmark datasets.

All datasets are \emph{map-matched}, meaning GPS points have been aligned to road segments via sophisticated algorithms. This preprocessing step, performed by domain experts, eliminates GPS noise and ensures trajectory realism. Our distillation framework operates entirely on these cleaned, road-level trajectories.


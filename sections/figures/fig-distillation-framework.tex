\begin{figure}[t]
    \centering
    \begin{tikzpicture}[
        % Layout structure:
        % - Input at origin (0,0)
        % - Vertical spacing: ~0.9cm between layers  
        % - Horizontal spacing: ±1.9cm for branches
        % - Left margin: -1.7cm for label path
        % - Temperature layer: yshift=3.9cm
        % Base styles
        node distance=0.9cm and 2.1cm,
        every node/.style={font=\small},
        % Box styles (professional colors)
        mainbox/.style={draw, rectangle, rounded corners=2pt, minimum width=2.5cm, minimum height=0.7cm, align=center, line width=0.5pt},
        frozen/.style={mainbox, fill=blue!12, draw=blue!60, dashed, line width=0.6pt},
        trainable/.style={mainbox, fill=green!12, draw=green!60},
        data/.style={mainbox, fill=gray!8, draw=gray!50},
        loss/.style={mainbox, fill=red!10, draw=red!50, minimum width=2cm},
        % Arrow styles - sleeker
        arrow/.style={-{Stealth[scale=0.6]}, semithick, draw=gray!60},
        flowArrow/.style={-{Stealth[scale=0.65]}, thick, draw=blue!50},
        ]

        % ===== MAIN PIPELINE (CENTER, BOTTOM TO TOP) =====
        
        % 1. Input (bottom)
        \node[data] (input) at (0, 0) {
            \textbf{Input Trajectory}
        };
        
        % 2. Vocabulary Mapping
        \node[data, above=0.9cm of input] (mapping) {
            \textbf{Road-Grid Mapping}
        };
        \draw[flowArrow] (input) -- (mapping);
        
        % 3a. Teacher model with winter icon
        \node[frozen, above left=2cm and 1.9cm of mapping, minimum width=2.8cm, minimum height=0.9cm] (teacher) {
            \includegraphics[height=0.35cm]{sections/figures/Icons8/icons8-winter-100.png}
            \hspace{0.1cm}$\mathcal{L}_\phi$\\
            {\footnotesize LM-TAD, Frozen}
        };
        
        % 3b. Student model with fire icon
        \node[trainable, above right=2cm and 1.9cm of mapping, minimum width=2.8cm, minimum height=0.9cm] (student) {
            \includegraphics[height=0.35cm]{sections/figures/Icons8/icons8-fire-100.png}
            \hspace{0.1cm}$\mathcal{H}_\theta$\\
            {\footnotesize HOSER, Trainable}
        };
        
        % 4. Arrows with data flow labels
        \draw[arrow] (mapping) -| node[pos=0.7, above, font=\footnotesize] {$\mathbf{z}_{1:t} \in \mathcal{Z}^t$} (teacher.south);
        \draw[arrow] (mapping) -| node[pos=0.7, above, font=\footnotesize] {$\mathcal{C}_t \subseteq \mathcal{V}$} (student.south);
        
        % Arrow from input to HOSER: Route FAR right around all boxes
        \coordinate (input_right) at ($(input.east)+(0.3,0)$);
        \coordinate (student_right) at ($(student.east)+(0.5,0)$);
        \draw[arrow] (input.east) -- (input_right) |- (student_right) -- (student.east);
        
        % 5. Temperature scaling - optimized spacing
        \node[data, above=1.1cm of mapping, yshift=3.9cm, minimum width=3cm] (temp) {
            \textbf{Temperature Scaling}
        };
        \draw[arrow] (teacher) -- ++(0,1.5) -| node[pos=0.15, above, font=\footnotesize] {$q^{(\tau)}$} (temp.west);
        \draw[arrow] (student) -- ++(0,1.5) -| node[pos=0.15, above, font=\footnotesize] {$\boldsymbol{\ell}^{\mathcal{H}}$} (temp.east);
        
        % 7. Multi-task losses
        \node[loss, above left=0.9cm and 1.3cm of temp] (ce) {
            \textbf{Cross-Entropy}
        };
        \node[loss, above=0.9cm of temp] (kl) {
            \textbf{KL Divergence}
        };
        \node[loss, above right=0.9cm and 1.3cm of temp] (time_loss) {
            \textbf{Time Loss}
        };
        
        \draw[arrow] (temp) -- (kl);
        
        % Input → Cross-Entropy: Route FAR left around all boxes with label
        \coordinate (input_left) at ($(input.west)+(-0.3,0)$);
        \coordinate (ce_left) at ($(ce.west)+(-0.5,0)$);
        \draw[arrow, dashed] (input.west) -- (input_left) |- (ce_left) node[pos=0.5, rotate=90, above, font=\footnotesize] {labels} -- (ce.west);
        
        % Logits → Time Loss: Route from student around to Time Loss
        \draw[arrow, dashed] (student.east) -- ++(0.8,0) |- node[pos=0.85, right, font=\footnotesize] {$\hat{t}$} (time_loss.east);
        
        % 8. Total loss
        \node[loss, above=0.9cm of kl, minimum width=4.5cm, minimum height=0.8cm] (total) {
            \textbf{Total Loss}\\
            {\footnotesize Weighted Combination}
        };
        \draw[arrow] (ce) |- (total.west);
        \draw[arrow] (kl) -- (total);
        \draw[arrow] (time_loss) |- (total.east);
        
        % 9. Gradient flow (top)
        \draw[flowArrow, red!60, line width=0.9pt] (total) -- ++(0, 0.8) node[above, font=\normalsize\bfseries] {Backprop to Student};

    \end{tikzpicture}
    \caption{Knowledge distillation framework with bottom-to-top information flow. The frozen teacher LM-TAD provides soft targets via temperature-scaled distributions over grid cells. The trainable student HOSER learns from hard labels (cross-entropy), auxiliary tasks (time prediction), and soft teacher knowledge (KL divergence). The road-grid mapping bridges the cross-vocabulary gap, enabling knowledge transfer despite architectural and vocabulary differences.}
    \label{fig:distillation-framework}
\end{figure}

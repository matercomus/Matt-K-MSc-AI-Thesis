\begin{figure}[t]
    \centering
    \begin{tikzpicture}[
        scale=0.8,
        transform shape,
        % Base styles
        node distance=0.9cm and 2cm,
        every node/.style={font=\footnotesize},
        % Box styles (professional colors)
        mainbox/.style={draw, rectangle, rounded corners=2pt, minimum width=2.2cm, minimum height=0.6cm, align=center, line width=0.5pt},
        smallbox/.style={draw, rectangle, rounded corners=1.5pt, minimum width=1.6cm, minimum height=0.45cm, align=center, line width=0.4pt},
        frozen/.style={mainbox, fill=blue!12, draw=blue!60, dashed, line width=0.6pt},
        trainable/.style={mainbox, fill=green!12, draw=green!60},
        data/.style={mainbox, fill=gray!8, draw=gray!50},
        loss/.style={mainbox, fill=red!10, draw=red!50, minimum width=1.8cm},
        magnify/.style={draw, rectangle, rounded corners=3pt, dashed, draw=orange!60, fill=orange!3, line width=0.5pt},
        % Arrow styles
        arrow/.style={-{Stealth[scale=0.6]}, thick, draw=gray!70},
        flowArrow/.style={-{Stealth[scale=0.7]}, very thick, draw=blue!60},
        dashed_connect/.style={dashed, draw=orange!50, line width=0.4pt},
        ]

        % ===== MAIN PIPELINE (CENTER, BOTTOM TO TOP) =====
        
        % 1. Input (bottom)
        \node[data] (input) at (0, 0) {
            \textbf{Input}\\
            {\tiny $\mathbf{r}_{1:t}$}
        };
        
        % 2. Vocabulary Mapping
        \node[data, above=0.9cm of input] (mapping) {
            \textbf{Mapping} $\psi$
        };
        \draw[flowArrow] (input) -- (mapping);
        
        % 3a. Grid sequence (left branch)
        \node[data, above left=0.9cm and 1.8cm of mapping, minimum width=2cm] (grid) {
            Grid\\
            {\tiny $\mathbf{z}_{1:t}$}
        };
        \draw[arrow] (mapping) -| (grid);
        
        % 3b. Candidate set (right branch)
        \node[data, above right=0.9cm and 1.8cm of mapping, minimum width=2cm] (candidates) {
            Candidates\\
            {\tiny $\mathcal{C}_t$}
        };
        \draw[arrow] (mapping) -| (candidates);
        
        % 4a. Teacher model
        \node[frozen, above=0.9cm of grid, minimum width=2.2cm] (teacher) {
            \textbf{Teacher}\\
            {\tiny \textit{Frozen}}
        };
        \draw[arrow] (grid) -- (teacher);
        
        % 4b. Student model
        \node[trainable, above=0.9cm of candidates, minimum width=2.2cm] (student) {
            \textbf{Student}\\
            {\tiny \textit{Trainable}}
        };
        \draw[arrow] (candidates) -- (student);
        % Arrow from input to student
        \draw[arrow] (input.east) -- ++(0.7,0) |- (student.south east);
        
        % 5a. Teacher output
        \node[data, above=0.9cm of teacher, minimum width=2cm] (teacher_out) {
            {\tiny $q_t$}
        };
        \draw[arrow] (teacher) -- (teacher_out);
        
        % 5b. Student output
        \node[data, above=0.9cm of student, minimum width=2cm] (student_out) {
            {\tiny $\ell_t$}
        };
        \draw[arrow] (student) -- (student_out);
        
        % 6. Temperature scaling
        \node[data, above=1.3cm of mapping, yshift=5cm, minimum width=2.5cm] (temp) {
            \textbf{Temp Scale} $\tau$
        };
        \draw[arrow] (teacher_out) |- (temp.west);
        \draw[arrow] (student_out) |- (temp.east);
        
        % 7. Multi-task losses
        \node[loss, above left=1.0cm and 1.2cm of temp] (ce) {
            $\mathcal{L}_{\text{CE}}$
        };
        \node[loss, above=1.0cm of temp] (kl) {
            $\mathcal{L}_{\text{KL}}$
        };
        \node[loss, above right=1.0cm and 1.2cm of temp] (time_loss) {
            $\mathcal{L}_{\text{time}}$
        };
        
        \draw[arrow] (temp) -- (kl);
        % Ground truth to CE
        \coordinate (gt_point) at ([xshift=-1.5cm]input.west);
        \draw[arrow, dashed] (input.west) -- (gt_point) |- (ce.south);
        % Student output to time loss  
        \draw[arrow, dashed] (student_out.east) -- ++(0.6,0) |- (time_loss.south);
        
        % 8. Total loss
        \node[loss, above=1.0cm of kl, minimum width=3.8cm, minimum height=0.75cm] (total) {
            \textbf{Total Loss}\\
            {\tiny $\mathcal{L}_{\text{CE}} + \alpha \mathcal{L}_{\text{time}} + \lambda \mathcal{L}_{\text{KL}}$}
        };
        \draw[arrow] (ce) |- (total.west);
        \draw[arrow] (kl) -- (total);
        \draw[arrow] (time_loss) |- (total.east);
        
        % 9. Gradient flow (top)
        \draw[flowArrow, red!70, line width=1pt] (total) -- ++(0, 0.6) node[above, font=\footnotesize\bfseries] {Backprop};
        
        
        % ===== LEFT MAGNIFICATION: VOCABULARY ALIGNMENT =====
        
        \begin{scope}[shift={(-5.5, 0.8)}]
            \node[magnify, minimum width=2.8cm, minimum height=3.2cm] (mag_vocab_box) at (0, 0) {};
            \node[above=0.03cm of mag_vocab_box.north, font=\footnotesize\bfseries, text=orange!70!black] {\tiny Vocab Alignment};
            
            % Road vocabulary
            \node[smallbox, fill=green!8, draw=green!50, font=\tiny] (roads) at (0, 0.85) {
                Roads\\
                {\scriptsize $|\mathcal{V}|$}
            };
            
            % Mapping arrow with psi
            \draw[-{Stealth[scale=0.5]}, thick] (roads) -- ++(0, -0.55) node[midway, right, font=\footnotesize] {$\psi$};
            
            % Grid cells
            \node[smallbox, fill=blue!8, draw=blue!50, font=\tiny] (cells) at (0, -0.15) {
                Grid\\
                {\scriptsize $|\mathcal{Z}|$}
            };
            
            % Visual grid representation (small 3x3 grid)
            \begin{scope}[shift={(0, -1.1)}, scale=0.1]
                \foreach \x in {0,1,2} {
                    \foreach \y in {0,1,2} {
                        \draw[draw=gray!40, fill=gray!5, line width=0.3pt] (\x, \y) rectangle (\x+1, \y+1);
                    }
                }
                % Highlight center cell
                \draw[draw=blue!60, fill=blue!15, line width=0.5pt] (1, 1) rectangle (2, 2);
            \end{scope}
            
            \node[font=\tiny, text width=2.4cm, align=center] at (0, -1.4) {
                Top-64 filtering
            };
        \end{scope}
        
        % Dashed line connecting magnification to main
        \draw[dashed_connect] (mag_vocab_box.east) -- (mapping.west);
        
        
        % ===== RIGHT MAGNIFICATION: MODEL ARCHITECTURES =====
        
        \begin{scope}[shift={(5.5, 2.5)}]
            \node[magnify, minimum width=2.8cm, minimum height=4.5cm] (mag_arch_box) at (0, 0) {};
            \node[above=0.03cm of mag_arch_box.north, font=\footnotesize\bfseries, text=orange!70!black] {\tiny Architectures};
            
            % Teacher architecture
            \node[font=\tiny\bfseries, text=blue!70] at (0, 1.8) {Teacher};
            \node[smallbox, fill=blue!8, draw=blue!50, font=\tiny, text width=2.2cm, minimum height=1.5cm] (teacher_arch) at (0, 0.8) {
                \textbf{LM-TAD}\\[1pt]
                {\scriptsize Transformer}\\
                {\scriptsize Grid input}\\
                {\scriptsize Frozen}
            };
            
            % Separator
            \draw[draw=gray!30] (-1.1, -0.15) -- (1.1, -0.15);
            
            % Student architecture  
            \node[font=\tiny\bfseries, text=green!70!black] at (0, -0.65) {Student};
            \node[smallbox, fill=green!8, draw=green!50, font=\tiny, text width=2.2cm, minimum height=1.5cm] (student_arch) at (0, -1.65) {
                \textbf{HOSER}\\[1pt]
                {\scriptsize GCN + Nav}\\
                {\scriptsize Road input}\\
                {\scriptsize Trainable}
            };
        \end{scope}
        
        % Dashed lines connecting magnification to main models
        \draw[dashed_connect] (mag_arch_box.west) -- (teacher.east);
        \draw[dashed_connect] (mag_arch_box.south west) -- (student.east);

    \end{tikzpicture}
    \caption{Knowledge distillation framework with bottom-to-top information flow. The frozen teacher $\mathcal{L}_\phi$ provides soft targets via temperature-scaled distributions over grid cells. The trainable student $\mathcal{H}_\theta$ learns from hard labels (cross-entropy), auxiliary tasks (time prediction), and soft teacher knowledge (KL divergence). The vocabulary mapping $\psi$ bridges the cross-vocabulary gap between road segments and grid cells, enabling knowledge transfer despite architectural and vocabulary differences.}
    \label{fig:distillation-framework}
\end{figure}

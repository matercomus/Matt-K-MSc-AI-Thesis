\begin{figure}[t]
    \centering
    \begin{tikzpicture}[
        scale=0.85,
        transform shape,
        % Base styles
        node distance=0.9cm and 2cm,
        every node/.style={font=\footnotesize},
        % Box styles (professional colors)
        mainbox/.style={draw, rectangle, rounded corners=2pt, minimum width=2.2cm, minimum height=0.6cm, align=center, line width=0.5pt},
        frozen/.style={mainbox, fill=blue!12, draw=blue!60, dashed, line width=0.6pt},
        trainable/.style={mainbox, fill=green!12, draw=green!60},
        data/.style={mainbox, fill=gray!8, draw=gray!50},
        loss/.style={mainbox, fill=red!10, draw=red!50, minimum width=1.8cm},
        % Arrow styles
        arrow/.style={-{Stealth[scale=0.6]}, thick, draw=gray!70},
        flowArrow/.style={-{Stealth[scale=0.7]}, very thick, draw=blue!60},
        ]

        % ===== MAIN PIPELINE (CENTER, BOTTOM TO TOP) =====
        
        % 1. Input (bottom)
        \node[data] (input) at (0, 0) {
            \textbf{Input}\\
            {\tiny Trajectory}
        };
        
        % 2. Vocabulary Mapping
        \node[data, above=0.9cm of input] (mapping) {
            \textbf{Mapping}\\
            {\tiny Road to Grid}
        };
        \draw[flowArrow] (input) -- (mapping);
        
        % 3a. Grid sequence (left branch)
        \node[data, above left=0.9cm and 1.8cm of mapping, minimum width=2cm] (grid) {
            \textbf{Grid}\\
            {\tiny Sequence}
        };
        \draw[arrow] (mapping) -| (grid);
        
        % 3b. Candidate set (right branch)
        \node[data, above right=0.9cm and 1.8cm of mapping, minimum width=2cm] (candidates) {
            \textbf{Candidates}\\
            {\tiny Top-64}
        };
        \draw[arrow] (mapping) -| (candidates);
        
        % 4a. Teacher model
        \node[frozen, above=0.9cm of grid, minimum width=2.5cm] (teacher) {
            \textbf{Teacher}\\
            {\tiny Transformer, Frozen}
        };
        \draw[arrow] (grid) -- (teacher);
        
        % 4b. Student model
        \node[trainable, above=0.9cm of candidates, minimum width=2.5cm] (student) {
            \textbf{Student}\\
            {\tiny GCN + Navigator, Trainable}
        };
        \draw[arrow] (candidates) -- (student);
        % Arrow from input to student
        \draw[arrow] (input.east) -- ++(0.7,0) |- (student.south east);
        
        % 5a. Teacher output
        \node[data, above=0.9cm of teacher, minimum width=2cm] (teacher_out) {
            \textbf{Teacher}\\
            {\tiny Probabilities}
        };
        \draw[arrow] (teacher) -- (teacher_out);
        
        % 5b. Student output
        \node[data, above=0.9cm of student, minimum width=2cm] (student_out) {
            \textbf{Student}\\
            {\tiny Logits}
        };
        \draw[arrow] (student) -- (student_out);
        
        % 6. Temperature scaling
        \node[data, above=1.3cm of mapping, yshift=5cm, minimum width=2.8cm] (temp) {
            \textbf{Temperature}\\
            {\tiny Scaling}
        };
        \draw[arrow] (teacher_out) |- (temp.west);
        \draw[arrow] (student_out) |- (temp.east);
        
        % 7. Multi-task losses
        \node[loss, above left=1.0cm and 1.2cm of temp] (ce) {
            \textbf{Cross}\\
            {\tiny Entropy}
        };
        \node[loss, above=1.0cm of temp] (kl) {
            \textbf{KL}\\
            {\tiny Divergence}
        };
        \node[loss, above right=1.0cm and 1.2cm of temp] (time_loss) {
            \textbf{Time}\\
            {\tiny MAPE}
        };
        
        \draw[arrow] (temp) -- (kl);
        % Ground truth to CE
        \coordinate (gt_point) at ([xshift=-1.5cm]input.west);
        \draw[arrow, dashed] (input.west) -- (gt_point) |- (ce.south) node[pos=0.85, left, font=\tiny] {labels};
        % Student output to time loss  
        \draw[arrow, dashed] (student_out.east) -- ++(0.6,0) |- (time_loss.south);
        
        % 8. Total loss
        \node[loss, above=1.0cm of kl, minimum width=4cm, minimum height=0.75cm] (total) {
            \textbf{Total Loss}\\
            {\tiny Weighted Combination}
        };
        \draw[arrow] (ce) |- (total.west);
        \draw[arrow] (kl) -- (total);
        \draw[arrow] (time_loss) |- (total.east);
        
        % 9. Gradient flow (top)
        \draw[flowArrow, red!70, line width=1pt] (total) -- ++(0, 0.7) node[above, font=\small\bfseries] {Backprop to Student};

    \end{tikzpicture}
    \caption{Knowledge distillation framework with bottom-to-top information flow. The frozen teacher $\mathcal{L}_\phi$ provides soft targets via temperature-scaled distributions over grid cells. The trainable student $\mathcal{H}_\theta$ learns from hard labels (cross-entropy), auxiliary tasks (time prediction), and soft teacher knowledge (KL divergence). The vocabulary mapping $\psi$ bridges the cross-vocabulary gap between road segments and grid cells, enabling knowledge transfer despite architectural and vocabulary differences.}
    \label{fig:distillation-framework}
\end{figure}

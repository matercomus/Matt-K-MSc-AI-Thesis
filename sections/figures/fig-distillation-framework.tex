\begin{figure}[t]
    \centering
    \begin{tikzpicture}[
        % Layout structure:
        % - Input at origin (0,0)
        % - Vertical spacing: ~0.9cm between layers  
        % - Horizontal spacing: ±1.9cm for branches
        % - Left margin: -1.7cm for label path
        % - Temperature layer: yshift=3.9cm
        % Base styles
        node distance=0.9cm and 2.1cm,
        every node/.style={font=\small},
        % Box styles (professional colors)
        mainbox/.style={draw, rectangle, rounded corners=2pt, minimum width=2.5cm, minimum height=0.7cm, align=center, line width=0.5pt},
        frozen/.style={mainbox, fill=blue!12, draw=blue!60, dashed, line width=0.6pt},
        trainable/.style={mainbox, fill=green!12, draw=green!60},
        data/.style={mainbox, fill=gray!8, draw=gray!50},
        loss/.style={mainbox, fill=red!10, draw=red!50, minimum width=2cm},
        % Arrow styles - sleeker
        arrow/.style={-{Stealth[scale=0.6]}, semithick, draw=gray!60},
        flowArrow/.style={-{Stealth[scale=0.65]}, thick, draw=blue!50},
        ]

        % ===== MAIN PIPELINE (CENTER, BOTTOM TO TOP) =====
        
        % 1. Input (bottom)
        \node[data] (input) at (0, 0) {
            \textbf{Input Trajectory}
        };
        
        % 2. Vocabulary Mapping
        \node[data, above=0.9cm of input] (mapping) {
            \textbf{Road-Grid Mapping}
        };
        \draw[flowArrow] (input) -- (mapping);
        
        % 3a. Grid sequence (left branch)
        \node[data, above left=0.9cm and 1.9cm of mapping, minimum width=2.3cm] (grid) {
            \textbf{Grid Sequence}
        };
        \draw[arrow] (mapping) -| (grid);
        
        % 3b. Candidate set (right branch)
        \node[data, above right=0.9cm and 1.9cm of mapping, minimum width=2.3cm] (candidates) {
            \textbf{Road Candidates}\\
            {\footnotesize (Top-64)}
        };
        \draw[arrow] (mapping) -| (candidates);
        
        % 4a. Teacher model with winter icon
        \node[frozen, above=0.9cm of grid, minimum width=2.8cm, minimum height=0.9cm] (teacher) {
            \includegraphics[height=0.35cm]{sections/figures/Icons8/icons8-winter-100.png}
            \hspace{0.1cm}\textbf{LM-TAD}\\
            {\footnotesize Transformer, Frozen}
        };
        \draw[arrow] (grid) -- (teacher);
        
        % 4b. Student model with fire icon
        \node[trainable, above=0.9cm of candidates, minimum width=2.8cm, minimum height=0.9cm] (student) {
            \includegraphics[height=0.35cm]{sections/figures/Icons8/icons8-fire-100.png}
            \hspace{0.1cm}\textbf{HOSER}\\
            {\footnotesize GCN + Navigator, Trainable}
        };
        \draw[arrow] (candidates) -- (student);
        
        % Arrow from input to HOSER: Route right to HOSER south edge
        \draw[arrow] (input) -| (student.south);
        
        % 5a. Teacher output
        \node[data, above=0.9cm of teacher, minimum width=2.3cm] (teacher_out) {
            \textbf{Probabilities}\\
            {\footnotesize (Teacher)}
        };
        \draw[arrow] (teacher) -- (teacher_out);
        
        % 5b. Student output
        \node[data, above=0.9cm of student, minimum width=2.3cm] (student_out) {
            \textbf{Logits}\\
            {\footnotesize (Student)}
        };
        \draw[arrow] (student) -- (student_out);
        
        % 6. Temperature scaling - optimized spacing
        \node[data, above=1.1cm of mapping, yshift=3.9cm, minimum width=3cm] (temp) {
            \textbf{Temperature Scaling}
        };
        \draw[arrow] (teacher_out) -| (temp.west);
        \draw[arrow] (student_out) -| (temp.east);
        
        % 7. Multi-task losses
        \node[loss, above left=0.9cm and 1.3cm of temp] (ce) {
            \textbf{Cross-Entropy}
        };
        \node[loss, above=0.9cm of temp] (kl) {
            \textbf{KL Divergence}
        };
        \node[loss, above right=0.9cm and 1.3cm of temp] (time_loss) {
            \textbf{Time Loss}
        };
        
        \draw[arrow] (temp) -- (kl);
        
        % Input → Cross-Entropy: Route left to CE south edge with label
        \draw[arrow, dashed] (input) -| node[pos=0.25, rotate=90, above, font=\footnotesize] {labels} (ce.south);
        
        % Logits → Time Loss: Route right to Time Loss east edge
        \draw[arrow, dashed] (student_out.east) -- ++(0.6,0) |- (time_loss.east);
        
        % 8. Total loss
        \node[loss, above=0.9cm of kl, minimum width=4.5cm, minimum height=0.8cm] (total) {
            \textbf{Total Loss}\\
            {\footnotesize Weighted Combination}
        };
        \draw[arrow] (ce) |- (total.west);
        \draw[arrow] (kl) -- (total);
        \draw[arrow] (time_loss) |- (total.east);
        
        % 9. Gradient flow (top)
        \draw[flowArrow, red!60, line width=0.9pt] (total) -- ++(0, 0.8) node[above, font=\normalsize\bfseries] {Backprop to Student};

    \end{tikzpicture}
    \caption{Knowledge distillation framework with bottom-to-top information flow. The frozen teacher LM-TAD provides soft targets via temperature-scaled distributions over grid cells. The trainable student HOSER learns from hard labels (cross-entropy), auxiliary tasks (time prediction), and soft teacher knowledge (KL divergence). The road-grid mapping bridges the cross-vocabulary gap, enabling knowledge transfer despite architectural and vocabulary differences.}
    \label{fig:distillation-framework}
\end{figure}

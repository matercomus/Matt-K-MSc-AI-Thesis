\section{Introduction}
\label{sec:introduction}

Urban taxi services have become increasingly important as cities grow more complex and public transportation networks struggle to serve all areas effectively. While taxis offer flexible, door-to-door transportation that fills critical gaps in urban mobility, they also present unique challenges that have gained significant attention in recent transportation research.

A particularly concerning issue in taxi operations is route inefficiency, where drivers deviate from optimal paths for various reasons. While some deviations can be justified by real-time traffic conditions or passenger preferences, others appear to stem from driver inexperience, navigation errors, or potentially deliberate route manipulation. These inefficiencies not only increase costs for passengers but also contribute to urban congestion and environmental impacts through unnecessary fuel consumption.

Machine learning approaches, particularly anomaly detection algorithms, have shown promise for identifying problematic routing patterns in transportation data. Traditional statistical methods can identify obvious deviations, but they often struggle with the contextual complexity of urban navigation decisions. Deep learning techniques offer better pattern recognition capabilities, yet they face practical limitations including the need for large labeled datasets and interpretability requirements for regulatory applications.

The development of effective anomaly detection systems for taxi route inefficiency faces a fundamental obstacle: the sensitive nature of location data severely limits access to real trajectory datasets for research purposes. This creates a critical research bottleneck where the very data needed to identify and prevent route manipulation remains inaccessible due to privacy concerns. Current privacy protection methods often destroy the subtle spatio-temporal patterns that anomaly detection algorithms need to function effectively, creating a paradox where stronger privacy measures undermine the utility of the data for legitimate research into transportation efficiency.

Synthetic data generation has emerged as a potential solution to this privacy-utility dilemma specifically for trajectory analysis. By creating artificial datasets that preserve essential statistical properties and anomaly characteristics of real taxi routes while protecting individual privacy, researchers could develop and evaluate route anomaly detection systems without compromising passenger confidentiality. However, trajectory data presents unique challenges for synthetic generation due to its complex spatio-temporal characteristics, urban context dependencies, and the critical need to preserve both normal routing patterns and subtle anomalous behavioral signatures that indicate route inefficiency.

This thesis proposes a novel framework for generating synthetic trajectory datasets that maintains the statistical and behavioral properties necessary for effective anomaly detection research while addressing critical privacy concerns. The approach focuses specifically on preserving the complex spatio-temporal patterns inherent in urban taxi operations, enabling privacy-preserving research and development in trajectory anomaly detection systems without requiring access to sensitive real-world data.


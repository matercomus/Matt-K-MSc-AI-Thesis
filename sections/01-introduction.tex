\section{Introduction}
\label{sec:introduction}

Accurate trajectory prediction is foundational to intelligent transportation systems, underpinning applications from dynamic navigation and fleet dispatch to digital-twin simulation of urban flow.  Modern cities contain tens of thousands of interconnected road segments; a practical predictor must therefore reason over large graphs
while delivering sub-second latency at metropolitan scale.

State-of-the-art transformer models excel at learning long-range spatial dependencies, yet their quadratic self-attention incurs inference times incompatible with real-time traffic management.  Conversely, lightweight graph-aware models such as\ HOSER achieve millisecond-level speed but fall short in route-completion accuracy.  This accuracy–latency dichotomy poses a central research challenge: \emph{how can one inherit the rich spatial knowledge of heavy models without deploying them at run time?}

This thesis answers the question by distilling the transformer-based LM-TAD anomaly detector into the hierarchical, low-latency HOSER predictor \emph{during training only}.  Our cross-task distillation transfers spatial priors learned in anomaly detection to next-step prediction, yielding a student that approaches transformer accuracy while preserving operational efficiency.

\paragraph{Contributions.}  We make four key contributions:
\begin{itemize}
  \item Propose the first cross-task distillation framework that transfers spatial knowledge from trajectory anomaly detection to trajectory prediction, enabling dramatic trajectory generation improvements on Beijing (85--89\% path completion vs vanilla's 12\%, 87--98\% JSD reduction) with no inference-time overhead.
  \item Introduce a scenario-level evaluation protocol with OD matching as the primary end-to-end metric, revealing context-dependent distillation behavior: universal benefits on complex urban networks (Beijing) versus spatially localized improvements in simpler environments (Porto).
  \item Demonstrate the validation--generation disconnect as a methodological insight: minimal validation accuracy gains (+0.01--0.23\%) produce dramatic generation quality improvements (+73\% OD match), motivating multi-objective optimization beyond next-step accuracy proxies.
  \item Establish the necessity of dataset-specific hyperparameter tuning through systematic Optuna-based optimization, revealing non-transferable optimal configurations across cities (4.3$\times$ divergence in $\lambda$, 42\% in $\tau$, 43\% in $w$ between Beijing and Porto) and highlighting the batch-size confound as a critical factor requiring controlled ablation studies.
\end{itemize}

\paragraph{Paper organisation.}  \autoref{sec:lit-review} surveys the evolution of trajectory modelling, culminating in the need for knowledge distillation.  \autoref{sec:methodology} details the LM-TAD\,$\rightarrow$\,HOSER distillation algorithm and training pipeline.  \autoref{sec:data-preprocessing} describes dataset preparation, and \autoref{sec:evaluation} presents empirical results.  We conclude with future research directions in \autoref{sec:conclusion}.


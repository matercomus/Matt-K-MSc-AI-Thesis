\documentclass[runningheads]{llncs}

% Package imports - organized by category
% Basic formatting and fonts
\usepackage{array}
\usepackage{fontspec}
\usepackage{graphicx}
\usepackage{amsmath}
\usepackage{enumitem}
\usepackage{xcolor}

% Tables and layout
\usepackage{tabularx}
\usepackage{longtable}

% URLs and hyperlinks
\usepackage{url}
\usepackage{hyperref}
\usepackage{color}

% Font configuration
\newfontfamily\chinesefont{FandolSong}

% Custom commands
\newcommand{\term}[1]{\textit{#1}}
\newcommand{\matt}[1]{{\bf\color{green!50!black}[#1]}} % Colored comments
\newcommand{\zh}[1]{{\chinesefont #1}} % Chinese font switch

% URL styling
\renewcommand\UrlFont{\color{blue}\rmfamily}
\urlstyle{rm}

\begin{document}

\input{title_page}
\title{Contribution Title}

\titlerunning{Abbreviated paper title}
% If the paper title is too long for the running head, you can set
% an abbreviated paper title here

\author{Mateusz Kędzia\inst{1}\orcidID{0009-0001-4296-4479}}

\authorrunning{Author et al.}
% First names are abbreviated in the running head.
% If there are more than two authors, 'et al.' is used.

\institute{Vrije Universiteit Amsterdam, Amsterdam\and
Vrije Universiteit Amsterdam, Amsterdam, The Netherlands
\email{lncs@springer.com}\\
\url{http://www.springer.com/gp/computer-science/lncs} \and
Vrije Universiteit Amsterdam, Amsterdam, The Netherlands\\
\email{\{abc,lncs\}@uni-heidelberg.de}}

\maketitle      % typeset the header of the contribution

\begin{abstract}
This study addresses the critical challenge of generating synthetic taxi trajectory datasets that preserve essential characteristics for anomaly detection research while ensuring passenger privacy protection. Urban taxi trajectory data contains sensitive location information that limits its availability for research purposes, creating a significant barrier to advancing anomaly detection methodologies. We propose a comprehensive framework for synthetic trajectory data generation that maintains statistical fidelity, behavioral patterns, and anomaly characteristics of real taxi routes while providing strong privacy guarantees.

Our approach leverages isolation forest analysis to understand normal and anomalous trajectory patterns in real data, extracting key statistical and behavioral properties that must be preserved in synthetic generation. The framework implements multiple privacy protection mechanisms including differential privacy, k-anonymity, and statistical aggregation to prevent inference of individual trajectories from synthetic data. Comprehensive evaluation demonstrates that synthetic datasets maintain the essential characteristics necessary for effective anomaly detection while providing strong privacy protection, enabling continued research advancement without compromising passenger confidentiality.

\keywords{Synthetic data generation \and Trajectory anomaly detection \and Privacy preservation \and Urban transportation \and Taxi routing}
\end{abstract}

\newpage

\section{Introduction}
\label{sec:introduction}

Urban transportation systems, particularly taxi services, play a crucial role in the mobility infrastructure of large cities worldwide. These services provide essential connectivity, filling gaps in public transportation networks and offering door-to-door convenience for millions of passengers daily. The efficiency and reliability of taxi operations directly impact urban traffic patterns, economic productivity, and citizen satisfaction with city services.

However, recent research has revealed significant inefficiencies in taxi route selection, with studies consistently showing that drivers often deviate from optimal paths. While some deviations may be justified by real-time traffic conditions or passenger preferences, others stem from more concerning causes including driver inexperience, deliberate route manipulation for fare maximization, or potentially malicious behavior. These routing anomalies not only affect passenger costs and travel times but also contribute to urban congestion and environmental impacts through unnecessary fuel consumption and emissions.

Artificial intelligence technologies, particularly machine learning approaches for anomaly detection, offer promising solutions for identifying and addressing these routing inefficiencies. Various AI methodologies have been developed to detect trajectory anomalies, ranging from classical statistical approaches to modern deep learning techniques. These methods can automatically identify suspicious route patterns that deviate significantly from normal driving behavior, enabling transportation authorities and taxi companies to implement corrective measures.

However, existing approaches face several critical limitations. Most traditional anomaly detection methods struggled with the complexity and contextual nature of urban routing decisions, often producing high false positive rates when applied to real-world taxi data. More sophisticated deep learning approaches, while achieving better accuracy, required extensive labeled datasets and lacked the interpretability necessary for practical deployment in regulatory contexts. Furthermore, the sensitive nature of location data raised significant privacy concerns, limiting the availability of real-world datasets for research and deployment.

Privacy-preserving techniques, particularly synthetic data generation, emerged as a promising solution to address these data availability and privacy constraints. By creating artificial datasets that preserve the statistical properties of real trajectory data while protecting individual privacy, synthetic data enabled the development and evaluation of anomaly detection systems without compromising passenger confidentiality. However, current synthetic data generation methods for trajectory data remained limited, particularly for capturing the complex spatial-temporal patterns inherent in urban taxi routes.

This study proposed a novel approach for generating synthetic taxi trajectory datasets that preserved the statistical and behavioral properties necessary for effective anomaly detection while addressing critical data privacy concerns. We focused specifically on creating realistic synthetic route data that maintained the complex spatial-temporal patterns inherent in urban taxi operations, enabling privacy-preserving research and development in trajectory anomaly detection systems.

The contribution of this work was threefold: (1) we developed an isolation forest methodology specifically adapted for urban taxi trajectory anomaly detection to establish ground truth patterns, (2) we introduced a comprehensive synthetic trajectory data generation framework that preserved both statistical properties and anomaly characteristics of real taxi routes, and (3) we provided extensive evaluation demonstrating that synthetic data maintained the essential characteristics necessary for effective anomaly detection while providing strong privacy guarantees.

\section{Literature Review}
\label{sec:literature-review}

Trajectory anomaly detection and privacy-preserving synthetic data generation evolved significantly due to increasing GPS data availability and privacy concerns. This review examined the progression from statistical approaches to machine learning methods, highlighting challenges in balancing detection accuracy, computational efficiency, and privacy protection.

\subsection{Route Anomaly Detection}
\label{sec:detection-review}

\paragraph{Statistical Approaches}
The first attempts at trajectory anomaly detection used simple statistical methods that compared individual trips against historical patterns. Wang et al.~\cite{wang2020statistical} developed a framework based on z-score normalization, examining how much each trip deviated from average duration, distance, and speed patterns. While this established a foundation for the field, the approach struggled with the fundamental challenge of distinguishing between legitimate route variations and truly suspicious behavior.

Chen and Liu~\cite{chen2021temporal} recognized that traffic patterns varied significantly by time and season, leading them to incorporate temporal context into statistical analysis. Their work highlighted a key insight: route anomalies cannot be understood without considering when they occur. However, even with temporal awareness, statistical methods remained limited by their reliance on simple thresholds and their inability to capture the complex, multidimensional nature of urban routing decisions.

\paragraph{Isolation-Based Approaches}
The limitations of threshold-based statistical methods motivated the development of isolation-based approaches. Zhang et al.~\cite{zhang2019ibat} introduced isolation forests to trajectory analysis, based on the principle that anomalies are easier to isolate than normal data points. Their iBat framework partitioned the feature space recursively, requiring fewer splits to isolate unusual routes compared to normal ones.

Li et al.~\cite{li2022enhanced} extended this concept with multi-scale analysis, recognizing that anomalies might occur at different levels - from individual route segments to complete trip patterns. Their approach incorporated contextual weighting, acknowledging that not all features are equally important for anomaly detection in different urban zones or time periods.

Despite these advances, isolation-based methods faced a critical limitation: they struggled to account for the legitimate variability inherent in urban transportation. Dynamic traffic conditions, construction zones, and passenger requests could all cause routes to appear anomalous when they were actually justified deviations.

\paragraph{Density-Based Methods}
To address the spatial-temporal complexity of trajectory data, researchers turned to density-based clustering methods. He et al.~\cite{he2020enhanced} developed enhanced DBSCAN techniques that used multiple distance metrics, including Dynamic Time Warping for temporal alignment and Hausdorff distances for spatial similarity. This approach recognized that trajectory anomalies must be understood in terms of both spatial deviation and temporal patterns.

Wu et al.~\cite{wu2023graph} further advanced this direction by representing trajectory relationships as graphs, enabling the modeling of complex interactions between routes that traditional clustering methods could not capture. The graph-based approach allowed for more nuanced understanding of how routes relate to each other in urban networks.

However, density-based methods introduced new challenges: high computational costs that limited real-time application, sensitivity to parameter settings that required expert tuning, and difficulty handling sparse regions where few similar routes existed for comparison.

\paragraph{Deep Learning Approaches}
The complexity of trajectory patterns motivated researchers to explore deep learning methods. Huang et al.~\cite{huang2021lstm} applied LSTM autoencoders with attention mechanisms, allowing the model to learn complex temporal dependencies and automatically focus on potentially anomalous trajectory segments. This approach represented a shift from hand-crafted features to learned representations.

Li et al.~\cite{li2023diffusion} introduced diffusion models that reframed anomaly detection as a reconstruction problem. By learning to generate normal trajectories, the model could identify anomalies as trajectories that were difficult to reconstruct accurately.

While deep learning approaches showed promise in handling complex patterns, they introduced new limitations: substantial computational requirements that hindered deployment, need for large labeled datasets that were difficult to obtain due to privacy concerns, and black-box characteristics that made results difficult to interpret for regulatory purposes.

\subsection{Privacy-Preserving Data Generation}
\label{sec:privacy-review}

The development of sophisticated anomaly detection methods was constrained by a fundamental challenge: the sensitive nature of trajectory data limited researchers' access to realistic datasets for algorithm development and evaluation.

\paragraph{Traditional Privacy Methods}
Early approaches attempted to balance privacy and utility through simple anonymization - removing identifiers and adding noise to coordinates. However, these methods proved inadequate when researchers demonstrated that trajectory patterns themselves could be used for re-identification, even without explicit identifiers.

\paragraph{Differential Privacy in Trajectory Data}
Zhang et al.~\cite{zhang2023differential} applied differential privacy principles to trajectory data, adding carefully calibrated noise to protect individual privacy while preserving aggregate patterns. Their approach addressed the mathematical requirements of privacy protection but faced the challenge of maintaining sufficient data utility for complex analytical tasks like anomaly detection.

The core tension became apparent: the subtle patterns that anomaly detection systems rely upon are precisely the types of information that privacy mechanisms tend to obscure. Stronger privacy guarantees often came at the cost of reduced utility for downstream applications.

\paragraph{k-Anonymity and Spatial Cloaking}
Liu et al.~\cite{liu2023enhanced} explored k-anonymity approaches that ensured each trajectory was indistinguishable from others in the dataset. While this provided some privacy protection, the method struggled with the high dimensionality of trajectory data and the difficulty of finding truly similar routes in sparse geographic regions.

\subsection{Synthetic Data Generation}
\label{sec:generation-review}

The privacy limitations of real trajectory data motivated researchers to explore synthetic data generation as an alternative approach that could provide both privacy protection and research utility.

\paragraph{Statistical Pattern Preservation}
Wang et al.~\cite{wang2023comprehensive} developed statistical models that captured aggregate patterns like origin-destination distributions and temporal trends using Gaussian mixture models and hidden Markov models. While this approach could generate realistic-looking trajectories that preserved basic statistical properties, it failed to capture the behavioral complexity underlying real routing decisions.

\paragraph{Behavioral Pattern Modeling}
Chen et al.~\cite{chen2023behavior} recognized that realistic trajectory generation required modeling driver behavior and decision-making processes. Their approach used reinforcement learning to simulate route choices and Bayesian networks to model behavioral factors. This represented a significant advancement in understanding that synthetic data must reflect not just statistical patterns but also the underlying human decisions that create those patterns.

\paragraph{Anomaly Pattern Generation}
Despite progress in generating normal trajectory patterns, a critical gap remained: existing methods focused almost exclusively on normal routing behavior. This created a fundamental limitation for anomaly detection research, as researchers lacked access to synthetic datasets that included the types of anomalous patterns necessary for robust system development and evaluation.

The challenge of generating realistic anomalies proved particularly difficult because anomalous behavior is inherently rare and diverse. Traditional generative models, optimized for majority patterns, struggled to capture and preserve the subtle characteristics that distinguish various types of trajectory anomalies from normal route variations.

\subsection{Research Gaps and Motivation}
\label{sec:gaps}

The evolution of trajectory anomaly detection and synthetic data generation revealed three fundamental challenges that motivated our research approach.

First, existing anomaly detection methods struggled to balance accuracy with practical deployment requirements. Statistical methods were interpretable but limited in handling complex patterns. Machine learning approaches could capture complexity but required large datasets and computational resources that were often unavailable due to privacy constraints.

Second, privacy-preserving techniques created a paradox: the more privacy protection was applied, the less useful the data became for anomaly detection research. Current approaches failed to maintain both strong privacy guarantees and the subtle patterns necessary for effective anomaly detection system development.

Third, and most critically, no existing synthetic data generation framework adequately addressed the challenge of preserving anomaly patterns while providing privacy protection. This gap severely limited researchers' ability to develop and evaluate robust anomaly detection systems without access to sensitive real-world data.

These limitations highlighted the need for a comprehensive approach that could generate synthetic trajectory data preserving both normal and anomalous patterns while providing strong privacy guarantees - precisely the gap that our work addresses.

\section{Methodology}
\label{sec:methodology}

\subsection{Isolation Forest for Trajectory Analysis}
\label{sec:iso}

\paragraph{Algorithm Implementation}

\paragraph{Key Adaptations for Trajectory Data}

\subsection{Statistical Pattern Extraction}
\label{sec:pattern-extraction}

\paragraph{Pattern Categories}

\textbf{Spatial Distributions}

\textbf{Temporal Patterns}

\textbf{Behavioral Characteristics}

\textbf{Anomaly Signatures}

\subsection{Enhanced Anomaly Detection}
\label{sec:improve}

\paragraph{Exception Handling Framework}

\textbf{Traffic-Induced Deviations}

\textbf{Passenger-Requested Deviations}

\textbf{Construction and Event Impacts}

\paragraph{Multi-Scale Analysis}

\subsection{Synthetic Trajectory Data Generation}
\label{sec:synthetic}

\paragraph{Generation Framework}

\paragraph{Privacy Preservation Mechanisms}

\paragraph{Quality Assurance Framework}

\section{Data and Preprocessing}
\label{sec:data-preprocessing}

\subsection{Dataset Description}
\label{sec:data}

The dataset used in this study consisted of Beijing taxi GPS data collected between 25.11.2019 and 01.12.2019. Each day contained approximately 16GB of raw GPS data, capturing the detailed movements of taxis throughout the metropolitan area. This large-scale dataset provided a rich source of real-world taxi routes for analysis and synthetic data generation.

\subsection{Data Preprocessing}
\label{sec:preprocessing}

\paragraph{Data Quality Issues Analysis}

\paragraph{Preprocessing Pipeline Implementation}

\paragraph{Quality Assessment Results}

\section{Experimental Setup and Results}
\label{sec:evaluation}

\subsection{Experimental Design}
\label{sec:exp-design}

\paragraph{Evaluation Phases}

\paragraph{Anomaly Detection Method Comparison}

\subsection{Anomaly Detection Results}
\label{sec:results}

\subsection{Synthetic Data Quality Evaluation}
\label{sec:synthetic-eval}

\paragraph{Statistical Fidelity Assessment}

\textbf{Distribution Comparisons}

\textbf{Statistical Test Results}

\paragraph{Anomaly Preservation Evaluation}

\textbf{Cross-Training Experiments}

\textbf{Detection Challenge Preservation}

\paragraph{Utility Validation}

\subsection{Privacy Preservation Assessment}
\label{sec:privacy-eval}

\paragraph{Attack Resistance Testing}

\textbf{Membership Inference Attacks}

\textbf{Trajectory Reconstruction Attacks}

\textbf{Location Privacy Protection}

\paragraph{Privacy-Utility Trade-off Analysis}

\subsection{Computational Performance Analysis}
\label{sec:performance}

\paragraph{Scalability Analysis}

\paragraph{Resource Requirements}

\section{Conclusion and Future Work}
\label{sec:conclusion}

\subsection{Research Contributions Summary}

\paragraph{Primary Contributions}

\subsection{Research Impact and Applications}

\paragraph{Academic Impact}

\paragraph{Practical Applications}

\subsection{Limitations and Challenges}

\paragraph{Current Limitations}

\paragraph{Technical Challenges}

\subsection{Future Research Directions}

\paragraph{Methodological Extensions}

\paragraph{Evaluation Framework Extensions}

\subsection{Concluding Remarks}

\newpage

% Bibliography
\bibliographystyle{splncs04}
\bibliography{references}

\matt{Fix Chinese chars not displaying}

\end{document}
% ===== Setup =====
\documentclass[runningheads]{llncs}

\usepackage{array}
\usepackage{graphicx}
\usepackage{amsmath}
\usepackage{enumitem}
\usepackage{xcolor}
\usepackage{tabularx}
\usepackage{longtable}
\usepackage{url}
\usepackage{hyperref}

% Font configuration (only if Chinese text is actually used)
% \usepackage{fontspec}
% \newfontfamily\chinesefont{FandolSong}

% Custom commands
% Custom Commands for Thesis Document
% =================================

% Text formatting commands
% ------------------------
\newcommand{\term}[1]{\textit{#1}}                                    % Italicize terms
\newcommand{\matt}[1]{{\bf\color{green!50!black}[#1]}}               % Colored comments for Matt

% Skeleton content commands for clean outline structure
% ----------------------------------------------------
\newcommand{\outline}[1]{{\small\color{gray!70}\textit{#1}}}         % Brief descriptions in gray italic
\newcommand{\todoheading}[1]{{\small\color{gray!60}$\triangleright$ \textit{#1}}} % Section headings with triangle markers
\newcommand{\skeletontext}[1]{\begin{quote}\small\color{gray!60}\textit{#1}\end{quote}} % Longer placeholder text in quote blocks

% Compact skeleton sections
\newcommand{\skeletonsection}[2]{
  \paragraph{#1} \outline{#2}
}

% Simplified compact outline environment
% Creates compact, gray-colored outline lists for skeleton content
\newenvironment{compactoutline}{
  \begin{quote}
    \small\color{gray!60}
    \begin{itemize}[leftmargin=1em,itemsep=0pt,parsep=0pt]
}{
    \end{itemize}
  \end{quote}
}

% Simple command for outline items - creates compact gray skeleton content entries
% Usage: \outlineitem{Topic Name -- Brief description}
\newcommand{\outlineitem}[1]{\item \todoheading{#1}} 

% URL styling
\renewcommand\UrlFont{\color{blue}\rmfamily}
\urlstyle{rm}


% ===== Document =====
\begin{document}

% ===== Title Page =====
\input{title_page}
\title{Contribution Title}
\titlerunning{Abbreviated paper title}
\author{Mateusz Kędzia\inst{1}\orcidID{0009-0001-4296-4479}}
\authorrunning{Kędzia M.G.}
\institute{Vrije Universiteit Amsterdam, Amsterdam}
\maketitle

\begin{abstract}
This study addresses the critical challenge of generating synthetic taxi trajectory datasets that preserve essential characteristics for anomaly detection research while ensuring passenger privacy protection. Urban taxi trajectory data contains sensitive location information that limits its availability for research purposes, creating a significant barrier to advancing anomaly detection methodologies. We propose a comprehensive framework for synthetic trajectory data generation that maintains statistical fidelity, behavioral patterns, and anomaly characteristics of real taxi routes while providing strong privacy guarantees.

Our approach leverages isolation forest analysis to understand normal and anomalous trajectory patterns in real data, extracting key statistical and behavioral properties that must be preserved in synthetic generation. The framework implements multiple privacy protection mechanisms including differential privacy, k-anonymity, and statistical aggregation to prevent inference of individual trajectories from synthetic data. Comprehensive evaluation demonstrates that synthetic datasets maintain the essential characteristics necessary for effective anomaly detection while providing strong privacy protection, enabling continued research advancement without compromising passenger confidentiality.

\keywords{Synthetic data generation \and Trajectory anomaly detection \and Privacy preservation \and Urban transportation \and Taxi routing}
\end{abstract}

\newpage

% ===== Content =====

\section{Introduction}
\label{sec:introduction}

Urban taxi services have become increasingly important as cities grow more complex and public transportation networks struggle to serve all areas effectively. While taxis offer flexible, door-to-door transportation that fills critical gaps in urban mobility, they also present unique challenges that have gained significant attention in recent transportation research.

A particularly concerning issue in taxi operations is route inefficiency, where drivers deviate from optimal paths for various reasons. While some deviations can be justified by real-time traffic conditions or passenger preferences, others appear to stem from driver inexperience, navigation errors, or potentially deliberate route manipulation. These inefficiencies not only increase costs for passengers but also contribute to urban congestion and environmental impacts through unnecessary fuel consumption.

Machine learning approaches, particularly anomaly detection algorithms, have shown promise for identifying problematic routing patterns in transportation data. Traditional statistical methods can identify obvious deviations, but they often struggle with the contextual complexity of urban navigation decisions. Deep learning techniques offer better pattern recognition capabilities, yet they face practical limitations including the need for large labeled datasets and interpretability requirements for regulatory applications.

The development of effective anomaly detection systems faces a fundamental obstacle: the sensitive nature of location data severely limits access to real trajectory datasets for research purposes. Current privacy protection methods often destroy the subtle patterns that anomaly detection algorithms need to function effectively, creating a paradox where stronger privacy measures can undermine the utility of the data for legitimate research.

Synthetic data generation has emerged as a potential solution to this privacy-utility dilemma. By creating artificial datasets that preserve essential statistical properties while protecting individual privacy, researchers could develop and evaluate anomaly detection systems without compromising passenger confidentiality. However, trajectory data presents unique challenges for synthetic generation due to its complex spatial-temporal characteristics and the need to preserve both normal and anomalous behavioral patterns.

This thesis proposes a novel framework for generating synthetic trajectory datasets that maintains the statistical and behavioral properties necessary for effective anomaly detection research while addressing critical privacy concerns. The approach focuses specifically on preserving the complex spatial-temporal patterns inherent in urban taxi operations, enabling privacy-preserving research and development in trajectory anomaly detection systems without requiring access to sensitive real-world data.

\section{Literature Review}
\label{sec:literature-review}

\begin{compactoutline}
   \outlineitem{Trajectory Anomaly Detection Methods}
   \outlineitem{Statistical Methods -- distance-based and density-based approaches, Z-score normalization~\cite{wang2020statistical}}
   \outlineitem{Clustering-Based Detection -- DBSCAN variations, enhanced density clustering~\cite{he2020enhanced}}
   \outlineitem{Isolation-Based Methods -- Isolation Forest, iBAT algorithm~\cite{zhang2019ibat}, multi-scale isolation~\cite{li2022enhanced}}
   \outlineitem{Deep Learning Approaches -- LSTM autoencoders~\cite{huang2021lstm}, diffusion models~\cite{li2023diffusion}}
   \outlineitem{Limitations -- parameter sensitivity, global threshold issues~\matt{ADD CITATION}}
   
   \outlineitem{Spatial-Temporal Pattern Analysis}
   \outlineitem{Trajectory Representation -- grid-based mapping, symbolic sequences~\matt{ADD CITATION}}
   \outlineitem{Temporal Context Analysis -- time-dependent behavior, traffic variability~\cite{chen2021temporal}}
   \outlineitem{Statistical Pattern Extraction -- origin-destination patterns, route densities~\matt{ADD CITATION}}
   \outlineitem{Distributed Processing -- Safety framework with spatial-feature mixing~\cite{wu2024safety}}
   
   \outlineitem{Privacy-Preserving Data Generation}
   \outlineitem{Differential Privacy -- trajectory noise injection~\cite{zhang2023differential}}
   \outlineitem{k-Anonymity Methods -- spatial cloaking, utility preservation~\cite{liu2023enhanced}}
   \outlineitem{Statistical Synthetic Generation -- pattern preservation frameworks~\cite{wang2023comprehensive}}
   \outlineitem{Behavior-Aware Generation -- reinforcement learning approaches~\cite{chen2023behavior}}
   \outlineitem{Privacy-Utility Trade-offs -- balancing protection and research utility~\matt{ADD CITATION}}
   
   \outlineitem{Anomaly Pattern Preservation}
   \outlineitem{Normal Pattern Generation -- statistical and behavioral modeling~\matt{ADD CITATION}}
   \outlineitem{Anomaly Pattern Generation -- rare event synthesis challenges~\matt{ADD CITATION}}
   \outlineitem{Pattern Fidelity Assessment -- maintaining detection characteristics~\matt{ADD CITATION}}
   
   \outlineitem{Evaluation and Validation}
   \outlineitem{Synthetic Data Quality Assessment -- statistical fidelity metrics~\matt{ADD CITATION}}
   \outlineitem{Anomaly Detection Performance -- precision, recall, F1-score evaluation~\matt{ADD CITATION}}
   \outlineitem{Privacy Protection Validation -- attack resistance testing~\matt{ADD CITATION}}
   
   \outlineitem{Research Gaps and Motivation}
   \outlineitem{Privacy Constraints -- limited real data access for research~\matt{ADD CITATION}}
   \outlineitem{Anomaly Pattern Preservation Gap -- maintaining detection characteristics in synthetic data~\matt{ADD CITATION}}
   \outlineitem{Comprehensive Framework Need -- integrated privacy-preserving anomaly detection~\matt{ADD CITATION}}
\end{compactoutline}


\section{Methodology}
\label{sec:methodology}

\subsection{Isolation Forest for Trajectory Analysis}
\label{sec:iso}

\begin{compactoutline}
  \outlineitem{Algorithm Implementation -- Core isolation forest adaptation for trajectory data}
  \outlineitem{Key Adaptations for Trajectory Data -- Feature engineering and distance metrics}
\end{compactoutline}

\subsection{Statistical Pattern Extraction}
\label{sec:pattern-extraction}

\begin{compactoutline}
  \outlineitem{Spatial Distributions -- Origin-destination patterns, route density maps}
  \outlineitem{Temporal Patterns -- Time-of-day effects, seasonal variations}
  \outlineitem{Behavioral Characteristics -- Driver decision patterns, route preferences}
  \outlineitem{Anomaly Signatures -- Characteristic patterns of anomalous behavior}
\end{compactoutline}

\subsection{Enhanced Anomaly Detection}
\label{sec:improve}

\begin{compactoutline}
  \outlineitem{Exception Handling Framework}
  \outlineitem{Traffic-Induced Deviations -- Real-time congestion handling}
  \outlineitem{Passenger-Requested Deviations -- Legitimate route changes}
  \outlineitem{Construction and Event Impacts -- Temporary route modifications}
  \outlineitem{Multi-Scale Analysis -- Segment-level vs. trip-level anomaly detection}
\end{compactoutline}

\subsection{Synthetic Trajectory Data Generation}
\label{sec:synthetic}

\begin{compactoutline}
  \outlineitem{Generation Framework -- Statistical model architecture and implementation}
  \outlineitem{Privacy Preservation Mechanisms -- Differential privacy, k-anonymity integration}
  \outlineitem{Quality Assurance Framework -- Validation metrics and testing procedures}
\end{compactoutline}

\section{Data and Preprocessing}
\label{sec:data-preprocessing}

\subsection{Dataset Description}
\label{sec:data}

The dataset used in this study consisted of Beijing taxi GPS data collected between 25.11.2019 and 01.12.2019. Each day contained approximately 16GB of raw GPS data, capturing the detailed movements of taxis throughout the metropolitan area. This large-scale dataset provided a rich source of real-world taxi routes for analysis and synthetic data generation.

\subsection{Data Preprocessing}
\label{sec:preprocessing}

\begin{compactoutline}
  \outlineitem{Data Quality Issues Analysis -- Missing data, GPS accuracy, temporal gaps}
  \outlineitem{Preprocessing Pipeline Implementation -- Cleaning, filtering, trajectory reconstruction}
  \outlineitem{Quality Assessment Results -- Statistics on data quality improvements}
\end{compactoutline}

\section{Experimental Setup and Results}
\label{sec:evaluation}

\subsection{Experimental Design}
\label{sec:exp-design}

\begin{compactoutline}
  \outlineitem{Evaluation Phases -- Real data analysis, synthetic generation, validation}
  \outlineitem{Anomaly Detection Method Comparison -- Baseline vs. proposed approach}
\end{compactoutline}

\subsection{Anomaly Detection Results}
\label{sec:results}

\skeletontext{Results from isolation forest analysis on real Beijing taxi data, including accuracy metrics, false positive rates, and comparison with baseline methods.}

\subsection{Synthetic Data Quality Evaluation}
\label{sec:synthetic-eval}

\begin{compactoutline}
  \outlineitem{Statistical Fidelity Assessment}
  \outlineitem{Distribution Comparisons -- Real vs. synthetic statistical properties}
  \outlineitem{Statistical Test Results -- Kolmogorov-Smirnov, Jensen-Shannon divergence}
  \outlineitem{Anomaly Preservation Evaluation}
  \outlineitem{Cross-Training Experiments -- Models trained on synthetic, tested on real}
  \outlineitem{Detection Challenge Preservation -- Maintaining difficulty of anomaly detection}
  \outlineitem{Utility Validation -- Performance of anomaly detection on synthetic data}
\end{compactoutline}

\subsection{Privacy Preservation Assessment}
\label{sec:privacy-eval}

\begin{compactoutline}
  \outlineitem{Attack Resistance Testing}
  \outlineitem{Membership Inference Attacks -- Can attackers identify original trajectories?}
  \outlineitem{Trajectory Reconstruction Attacks -- Ability to reconstruct individual routes}
  \outlineitem{Location Privacy Protection -- Geographic anonymization effectiveness}
  \outlineitem{Privacy-Utility Trade-off Analysis -- Quantitative analysis of privacy vs. utility}
\end{compactoutline}

\subsection{Computational Performance Analysis}
\label{sec:performance}

\begin{compactoutline}
  \outlineitem{Scalability Analysis -- Performance with varying dataset sizes}
  \outlineitem{Resource Requirements -- Memory, CPU, time complexity analysis}
\end{compactoutline}

\section{Conclusion and Future Work}
\label{sec:conclusion}

\subsection{Research Contributions Summary}

\begin{compactoutline}
  \outlineitem{Primary Contributions -- Novel synthetic generation framework, privacy-preserving anomaly detection}
\end{compactoutline}

\subsection{Research Impact and Applications}

\begin{compactoutline}
  \outlineitem{Academic Impact -- Contributions to trajectory analysis and privacy research}
  \outlineitem{Practical Applications -- Urban transportation, ride-sharing platforms}
\end{compactoutline}

\subsection{Limitations and Challenges}

\begin{compactoutline}
  \outlineitem{Current Limitations -- Computational complexity, geographical specificity}
  \outlineitem{Technical Challenges -- Privacy-utility trade-offs, scalability issues}
\end{compactoutline}

\subsection{Future Research Directions}

\begin{compactoutline}
  \outlineitem{Methodological Extensions -- Advanced generative models, multi-modal data}
  \outlineitem{Evaluation Framework Extensions -- Additional privacy metrics, real-world validation}
\end{compactoutline}

\subsection{Concluding Remarks}

\skeletontext{Summary of the research significance, implications for urban transportation research, and the potential for practical deployment of privacy-preserving trajectory anomaly detection systems.}

\newpage

% ===== Bibliography =====
\bibliographystyle{splncs04}
\bibliography{references}

% ===== Appendix =====
\appendix
\section{Appendix}
\label{sec:appendix}

\subsection{Appendix Section}
\label{sec:appendix-section}

\subsection{Appendix Section}
\label{sec:appendix-section-2}

\end{document}
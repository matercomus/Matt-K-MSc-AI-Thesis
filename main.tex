% ===== Setup =====
\documentclass[runningheads]{llncs}

\usepackage[utf8]{inputenc}
\usepackage[T1]{fontenc}
\usepackage{array}
\usepackage{graphicx}
\usepackage{amsmath}
\usepackage{enumitem}
\usepackage{xcolor}
\usepackage{tabularx}
\usepackage{longtable}
\usepackage{url}
\usepackage{hyperref}
\usepackage{algorithm}
\usepackage{algpseudocode}

% Font configuration (only if Chinese text is actually used)
% \usepackage{fontspec}
% \newfontfamily\chinesefont{FandolSong}

% Custom commands
% Custom Commands for Thesis Document
% =================================

% Text formatting commands
% ------------------------
\newcommand{\term}[1]{\textit{#1}}                                    % Italicize terms
\newcommand{\matt}[1]{{\bf\color{green!50!black}[#1]}}               % Colored comments for Matt

% Skeleton content commands for clean outline structure
% ----------------------------------------------------
\newcommand{\outline}[1]{{\small\color{gray!70}\textit{#1}}}         % Brief descriptions in gray italic
\newcommand{\todoheading}[1]{{\small\color{gray!60}$\triangleright$ \textit{#1}}} % Section headings with triangle markers
\newcommand{\skeletontext}[1]{\begin{quote}\small\color{gray!60}\textit{#1}\end{quote}} % Longer placeholder text in quote blocks

% Compact skeleton sections
\newcommand{\skeletonsection}[2]{
  \paragraph{#1} \outline{#2}
}

% Simplified compact outline environment
% Creates compact, gray-colored outline lists for skeleton content
\newenvironment{compactoutline}{
  \begin{quote}
    \small\color{gray!60}
    \begin{itemize}[leftmargin=1em,itemsep=0pt,parsep=0pt]
}{
    \end{itemize}
  \end{quote}
}

% Simple command for outline items - creates compact gray skeleton content entries
% Usage: \outlineitem{Topic Name -- Brief description}
\newcommand{\outlineitem}[1]{\item \todoheading{#1}} 

% URL styling
\renewcommand\UrlFont{\color{blue}\rmfamily}
\urlstyle{rm}


% ===== Document =====
\begin{document}

% ===== Title Page =====
\input{title_page}
\title{Knowledge Distillation for Map-Matched Trajectory Prediction: Improving Urban Route Prediction through Cross-Task Knowledge Transfer}
\titlerunning{Knowledge Distillation for Trajectory Prediction}
\author{Mateusz K{\k e}dzia\inst{1}\orcidID{0009-0001-4296-4479}}
\authorrunning{K{\k e}dzia M.G.}
\institute{Vrije Universiteit Amsterdam, Amsterdam}
\maketitle

\begin{abstract}
  Urban traffic management, transportation planning, and intelligent city systems require accurate real-time trajectory prediction to support policy decisions and optimize traffic flow. However, existing fast prediction models suffer from poor route completion rates (12-18\%), limiting their practical deployment for traffic regulators and urban planners. While sophisticated models like LM-TAD achieve superior spatial reasoning, their computational overhead (~3.4ms per trajectory vs ~0.1ms for fast models) prevents real-time application in city-wide traffic management systems, digital twin platforms, and large-scale simulations.

  This thesis addresses this challenge through training-time knowledge distillation, transferring spatial understanding from LM-TAD (a trajectory anomaly detection model) to HOSER (a fast zone-based prediction model). We demonstrate that repurposing the ``normal trajectory'' knowledge learned by anomaly detection models enables dramatic improvements in route prediction without inference-time overhead. Our distillation framework achieves 85-89\% path completion success (47-74$\times$ improvement over vanilla baseline), 87\% better distance distribution matching, and 98\% better spatial pattern fidelity on Beijing's 40,060-road network with 629,380 training trajectories. Hyperparameter optimization reveals that minimal distillation weight ($\lambda$=0.0014) with high temperature ($\tau$=4.37) enables effective knowledge transfer while preserving the student model's fast inference speed.

  The resulting system enables practical deployment for policy makers and traffic regulators, supporting applications in real-time traffic signal optimization, infrastructure planning, urban digital twins, agent-based traffic simulation, and high-quality synthetic trajectory data generation for training other models. This work demonstrates the viability of cross-task knowledge distillation for trajectory prediction and provides a scalable framework for integrating AI-based route prediction into operational traffic management systems.

  \keywords{Knowledge distillation \and Trajectory prediction \and Urban transportation \and Traffic management \and Digital twins \and Deep learning}
\end{abstract}

\newpage

% ===== Content =====

\section{Introduction}
\label{sec:introduction}

Accurate trajectory prediction is foundational to intelligent transportation systems, underpinning applications from dynamic navigation and fleet dispatch to digital-twin simulation of urban flow.  Modern cities contain tens of thousands of interconnected road segments; a practical predictor must therefore reason over large graphs
while delivering sub-second latency at metropolitan scale.

State-of-the-art transformer models excel at learning long-range spatial dependencies, yet their quadratic self-attention incurs inference times incompatible with real-time traffic management.  Conversely, lightweight graph-aware models such as\ HOSER achieve millisecond-level speed but fall short in route-completion accuracy.  This accuracy–latency dichotomy poses a central research challenge: \emph{how can one inherit the rich spatial knowledge of heavy models without deploying them at run time?}

This thesis answers the question by distilling the transformer-based LM-TAD anomaly detector into the hierarchical, low-latency HOSER predictor \emph{during training only}.  Our cross-task distillation transfers spatial priors learned in anomaly detection to next-step prediction, yielding a student that approaches transformer accuracy while preserving operational efficiency.

\paragraph{Contributions.}  We make four key contributions:
\begin{itemize}
  \item Propose the first cross-task distillation framework that transfers spatial knowledge from trajectory anomaly detection to trajectory prediction, enabling dramatic trajectory generation improvements on Beijing (85--89\% path completion vs vanilla's 12\%, 87--98\% JSD reduction) with no inference-time overhead.
  \item Introduce a scenario-level evaluation protocol with OD matching as the primary end-to-end metric, revealing context-dependent distillation behavior: universal benefits on complex urban networks (Beijing) versus spatially localized improvements in simpler environments (Porto).
  \item Demonstrate the validation--generation disconnect as a methodological insight: minimal validation accuracy gains (+0.01--0.23\%) produce dramatic generation quality improvements (+73\% OD match), motivating multi-objective optimization beyond next-step accuracy proxies.
  \item Establish the necessity of dataset-specific hyperparameter tuning through systematic Optuna-based optimization, revealing non-transferable optimal configurations across cities (4.3$\times$ divergence in $\lambda$, 42\% in $\tau$, 43\% in $w$ between Beijing and Porto) and highlighting the batch-size confound as a critical factor requiring controlled ablation studies.
\end{itemize}

\paragraph{Paper organisation.}  \autoref{sec:lit-review} surveys the evolution of trajectory modelling, culminating in the need for knowledge distillation.  \autoref{sec:methodology} details the LM-TAD\,$\rightarrow$\,HOSER distillation algorithm and training pipeline.  \autoref{sec:data-preprocessing} describes dataset preparation, and \autoref{sec:evaluation} presents empirical results.  We conclude with future research directions in \autoref{sec:conclusion}.



\section{Related Work}
\label{sec:lit-review}

This section reviews the key research areas relevant to knowledge distillation for trajectory prediction. We examine trajectory prediction and generation methods, synthetic data applications for urban systems, trajectory anomaly detection approaches that learn spatial patterns, knowledge distillation techniques, the architectural foundations in graph neural networks and transformers, and cross-task knowledge transfer methods.

% ===== BEGIN REWRITTEN BODY (chronological narrative) =====

% Overview paragraph
The evolution of data‐driven trajectory research mirrors the broader progression of sequential modelling in artificial intelligence.  Beginning with recurrent neural networks and hand‐crafted statistical baselines, the field has steadily adopted richer spatial representations, more expressive sequence architectures and, most recently, cross–task knowledge transfer.  The subsections below follow this chronological arc, critiquing the advantages and limitations of each paradigm to motivate the distillation framework presented in~\autoref{sec:methodology}.

\subsection{Classical Trajectory Prediction Models}
\label{sec:lit-traj-pred}
Early work framed next–location prediction as a sequence modelling problem amenable to recurrent neural networks (RNNs) and their gated variants.  Memory‐augmented LSTMs~\cite{liang2018memory} and variational autoencoders~\cite{liuOnlineAnomalousTrajectory2020} captured short-range dependencies and route uncertainty, while GAN‐based approaches such as LSTM-TrajGAN~\cite{raoLSTMTrajGANDeepLearning2020} attempted adversarially faithful path synthesis.  These models established the feasibility of learning spatial–temporal patterns directly from GPS data, but their limited receptive field and difficulty handling map constraints curbed real-world adoption.  \textit{Takeaway}: classical RNN/GAN models prove the concept yet struggle with long-range coherence and graph topology.

\subsection{Graph Neural Networks for Road Networks}
\label{sec:lit-gnn}
The introduction of graph neural networks (GNNs)~\cite{kipfSemisupervisedClassificationGraph2017,veličkovićGraphAttentionNetworks2018} provided an explicit inductive bias for road topology.  By representing intersections and road segments as nodes and edges, GCNs aggregated neighbourhood information, whereas GATs learned edge-specific attention weights, supporting fine-grained routing decisions.  Hierarchical hybrids further combined local and regional reasoning.  Despite clear spatial benefits, pure GNN solutions often incurred high inference latency on large urban graphs.  \textit{Takeaway}: GNNs embed topology elegantly but computational cost motivates search for lighter yet expressive alternatives.

\subsection{Transformer Architectures for Mobility Sequences}
\label{sec:lit-transformer}
Transformers~\cite{vaswaniAttentionAllYou2023} revolutionised sequential learning through self-attention.  Mobility research quickly adopted this paradigm—LM-TAD treats trajectories as token sequences, achieving state-of-the-art anomaly detection accuracy.  Large language model (LLM) adaptations such as PathGen-LLM~\cite{liPathGenLLMLargeLanguage} demonstrated zero-shot path generalisation.  However, the quadratic cost of self-attention renders vanilla transformers impractical for real-time traffic services.  \textit{Takeaway}: transformers learn rich global dependencies but impose prohibitive inference overhead.

\subsection{Deep Generative Approaches: VAEs to Diffusion and LLMs}
\label{sec:lit-generative}
Parallel to architectural advances, generative modelling transitioned from VAEs to diffusion processes and LLM-style decoders.  Diffusion-based TrajGDM~\cite{chuSimulatingHumanMobility2024} reframed generation as uncertainty reduction, yielding diverse and realistic paths.  TrajGPT~\cite{hsuTrajGPTControlledSynthetic2024} leveraged transformers for controllable synthesis, while foundation-scale models integrated multi-modal context~\cite{maLearningUniversalHuman2025}.  These methods improved fidelity but further increased computational and data demands.  \textit{Takeaway}: modern generative models capture complex mobility patterns yet exacerbate scalability concerns.

% ---- Road-so-far synthesis ----
\paragraph{Road so far.}  Classical RNNs established learnability, GNNs injected topology awareness, transformers unlocked long-range context, and diffusion/LLM generators raised realism.  Nevertheless, none reconcile spatial expressiveness with the latency constraints of traffic operations.

\subsection{Synthetic Trajectory Generation for Urban Applications}
\label{sec:lit-synthetic-urban}
Recent literature positions synthetic data as a utility-centric asset for simulation and policy rather than solely for privacy.  SynMob~\cite{zhuSynMobCreatingHighFidelity} and related frameworks retain geo-statistical properties critical to urban planning.  This shift underlines the importance of high-quality generation that scales across cities, a requirement echoed by foundation mobility models~\cite{maLearningUniversalHuman2025}.  \textit{Takeaway}: urban stakeholders demand scalable, high-fidelity synthesis, heightening the need for efficient yet accurate predictors.

\subsection{Trajectory Anomaly Detection and Spatial Learning}
\label{sec:lit-anomaly-spatial}
Anomaly detectors such as LM-TAD learn what \emph{normal} mobility looks like by modelling probability distributions over location tokens~\cite{heSpatiotemporalTrajectoryAnomaly2022,kongMobileTrajectoryAnomaly2024}.  Their spatial insight is invaluable, yet transformer-based detectors are slower than prediction-oriented models like HOSER.  \textit{Takeaway}: anomaly detection encodes rich spatial priors that remain untapped by fast predictors.

\subsection{Knowledge Distillation and Model Compression}
\label{sec:lit-distill}
Knowledge distillation~\cite{hintonDistillingKnowledgeNeural2015} addresses the accuracy–efficiency dilemma by transferring soft targets from a high-capacity teacher to a lightweight student.  Vision and NLP studies show that students can approximate teachers with negligible runtime overhead.  \textit{Takeaway}: distillation offers a principled route to inherit transformer knowledge without paying inference cost.

\subsection{Cross-Task Transfer and Foundation Mobility Models}
\label{sec:lit-transfer}
Cross-task transfer extends distillation across objectives—e.g.
leveraging anomaly-detection priors to boost prediction~\cite{maLearningUniversalHuman2025}.  Foundation mobility models exemplify multi-domain knowledge sharing, yet concrete methods for teacher–student bridging remain under-explored.  \textit{Takeaway}: transferring spatial knowledge across tasks is promising but lacks systematic methodologies.

% ---- Second Road-so-far synthesis ----
\paragraph{Road so far.}  The literature converges on two complementary insights: (i) transformers learn superior spatial representations, and (ii) operational systems require millisecond-scale inference.

\subsection*{Synthesis and Motivation for This Thesis}
\label{sec:lit-synthesis}
Bridging these insights, our work distils the transformer-based LM-TAD anomaly detector into the graph-aware, low-latency HOSER predictor during training only, as detailed in~\autoref{sec:methodology}. This approach inherits rich spatial priors while preserving real-time performance, directly addressing the shortcomings identified above.

% ===== END REWRITTEN BODY =====


\section{Methodology}
\label{sec:methodology}

This section presents our knowledge distillation framework for transferring spatial reasoning from a trajectory anomaly detector to a fast route predictor. Figure~\ref{fig:distillation-framework} illustrates the complete pipeline, showing how the frozen teacher and trainable student interact through vocabulary mapping and temperature-scaled distributions. We detail the vocabulary alignment mechanism (\autoref{sec:method-vocab}), the temperature-scaled distillation objective (\autoref{sec:method-kl}), the optimized training pipeline (\autoref{sec:method-training}), and the inference procedure (\autoref{sec:method-inference}).

\subsection{Preliminaries}
\label{sec:method-prelim}

\subsubsection{Notation}
Table~\ref{tab:notation} summarizes the mathematical notation used throughout this section.

\begin{table}[h]
    \centering
    \caption{Mathematical notation}
    \label{tab:notation}
    \begin{tabular}{ll}
        \toprule
        \textbf{Symbol}                                                                    & \textbf{Description}                             \\
        \midrule
        $\mathcal{G} = (\mathcal{V}, \mathcal{E})$                                         & Road network graph with $|\mathcal{V}|$ segments \\
        $\mathbf{r}_{1:t} = (r_1, \ldots, r_t)$                                            & Partial trajectory of $t$ road segments          \\
        $\mathcal{C}_t \subseteq \mathcal{V}$                                              & Candidate set at timestep $t$ (top-$k$ roads)    \\
        $\mathcal{Z} = \{1, \ldots, V\}$                                                   & Grid cell vocabulary             \\
        \midrule
        $\mathcal{L}_\phi : \mathcal{Z}^w \to \Delta^{|\mathcal{Z}|}$                      & Teacher model (frozen)                           \\
        $\mathcal{H}_\theta : \mathcal{V}^t \times \mathcal{V} \to \Delta^{|\mathcal{C}|}$ & Student model (trainable)                        \\
        $\psi : \mathcal{V} \to \mathcal{Z}$                                               & Cross-vocabulary mapping                         \\
        $q^{(\tau)}, p^{(\tau)}$                                                           & Temperature-scaled distributions                 \\
        $\tau$                                                                             & Temperature parameter                            \\
        $\lambda$                                                                          & Distillation loss weight                         \\
        \bottomrule
    \end{tabular}
\end{table}

\subsubsection{Problem Definition}
\label{sec:method-problem}

\begin{definition}[Trajectory]
    \label{def:trajectory}
    A \emph{trajectory} $T$ is a sequence of road segments:
    \[
        T = \{r_1, r_2, \ldots, r_n\} \quad \text{where } r_i \in \mathcal{V}
    \]
    representing a path through the road network $\mathcal{G} = (\mathcal{V}, \mathcal{E})$.
\end{definition}

\begin{definition}[Timestamped Trajectory]
    \label{def:trajectory-timestamped}
    A \emph{timestamped trajectory} $T$ includes arrival times at each road segment:
    \[
        T = \{(r_1, t_1), (r_2, t_2), \ldots, (r_n, t_n)\} \quad \text{where } r_i \in \mathcal{V}, t_i \in \mathbb{R}_+
    \]
    with timestamps satisfying $t_1 < t_2 < \cdots < t_n$.
\end{definition}

\begin{definition}[Cross-Vocabulary Mapping]
    \label{def:vocab-mapping}
    The mapping $\psi : \mathcal{V} \to \mathcal{Z}$ assigns each road segment to its containing grid cell:
    \begin{equation}
        \psi(v) = \left\lfloor \frac{x_v - x_{\min}}{\Delta_x} \right\rfloor \cdot n_{\text{cols}} +
        \left\lfloor \frac{y_v - y_{\min}}{\Delta_y} \right\rfloor
    \end{equation}
    where $(x_v, y_v) = \text{centroid}(v)$ and $\Delta_x, \Delta_y$ are grid resolutions.
\end{definition}

\begin{definition}[Cross-Task Knowledge Distillation]
    \label{def:distillation-problem}
    Given frozen teacher $\mathcal{L}_\phi$ trained for anomaly detection and trajectory dataset $\mathcal{D} = \{(\mathbf{r}_i, y_i)\}_{i=1}^N$, learn student parameters:
    \begin{equation}
        \theta^* = \arg\min_{\theta \in \Theta} \mathbb{E}_{(\mathbf{r}, y) \sim \mathcal{D}}
        \left[ \mathcal{L}_{\text{CE}}(\mathcal{H}_\theta(\mathbf{r}), y) +
        \lambda \mathcal{L}_{\text{KL}}^{(\tau)}(\mathcal{L}_\phi \circ \psi(\mathbf{r}), \mathcal{H}_\theta(\mathbf{r})) \right]
    \end{equation}
\end{definition}

\subsubsection{Model Specifications}
\label{sec:method-models}

We transfer knowledge from the pre-trained LM-TAD teacher $\mathcal{L}_\phi$~\cite{mbuyaTrajectoryAnomalyDetection2024} to the HOSER student $\mathcal{H}_\theta$~\cite{caoHolisticSemanticRepresentation2025}. The teacher operates on grid cell sequences with vocabulary size $|\mathcal{Z}| = V$, while the student predicts over $|\mathcal{V}|$ road segments.

\begin{figure}[t]
    \centering
    \begin{tikzpicture}[
        % Layout structure:
        % - Input at origin (0,0)
        % - Vertical spacing: ~0.9cm between layers  
        % - Horizontal spacing: ±1.9cm for branches
        % - Left margin: -1.7cm for label path
        % - Temperature layer: yshift=3.9cm
        % Base styles
        node distance=0.9cm and 2.1cm,
        every node/.style={font=\small},
        % Box styles (professional colors)
        mainbox/.style={draw, rectangle, rounded corners=2pt, minimum width=2.5cm, minimum height=0.7cm, align=center, line width=0.5pt},
        frozen/.style={mainbox, fill=blue!12, draw=blue!60, dashed, line width=0.6pt},
        trainable/.style={mainbox, fill=green!12, draw=green!60},
        data/.style={mainbox, fill=gray!8, draw=gray!50},
        loss/.style={mainbox, fill=red!10, draw=red!50, minimum width=2cm},
        % Arrow styles - sleeker
        arrow/.style={-{Stealth[scale=0.6]}, semithick, draw=gray!60},
        flowArrow/.style={-{Stealth[scale=0.65]}, thick, draw=blue!50},
        ]

        % ===== MAIN PIPELINE (CENTER, BOTTOM TO TOP) =====
        
        % 1. Input (bottom)
        \node[data] (input) at (0, 0) {
            \textbf{Input Trajectory}
        };
        
        % 2. Vocabulary Mapping
        \node[data, above=0.9cm of input] (mapping) {
            \textbf{Road-Grid Mapping}
        };
        \draw[flowArrow] (input) -- (mapping);
        
        % 3a. Grid sequence (left branch)
        \node[data, above left=0.9cm and 1.9cm of mapping, minimum width=2.3cm] (grid) {
            \textbf{Grid Sequence}
        };
        \draw[arrow] (mapping) -| (grid);
        
        % 3b. Candidate set (right branch)
        \node[data, above right=0.9cm and 1.9cm of mapping, minimum width=2.3cm] (candidates) {
            \textbf{Road Candidates}\\
            {\footnotesize (Top-64)}
        };
        \draw[arrow] (mapping) -| (candidates);
        
        % 4a. Teacher model with winter icon
        \node[frozen, above=0.9cm of grid, minimum width=2.8cm, minimum height=0.9cm] (teacher) {
            \includegraphics[height=0.35cm]{sections/figures/Icons8/icons8-winter-100.png}
            \hspace{0.1cm}\textbf{LM-TAD}\\
            {\footnotesize Transformer, Frozen}
        };
        \draw[arrow] (grid) -- (teacher);
        
        % 4b. Student model with fire icon
        \node[trainable, above=0.9cm of candidates, minimum width=2.8cm, minimum height=0.9cm] (student) {
            \includegraphics[height=0.35cm]{sections/figures/Icons8/icons8-fire-100.png}
            \hspace{0.1cm}\textbf{HOSER}\\
            {\footnotesize GCN + Navigator, Trainable}
        };
        \draw[arrow] (candidates) -- (student);
        
        % Arrow from input to HOSER: Route right to HOSER south edge
        \draw[arrow] (input) -| (student.south);
        
        % 5a. Teacher output
        \node[data, above=0.9cm of teacher, minimum width=2.3cm] (teacher_out) {
            \textbf{Probabilities}\\
            {\footnotesize (Teacher)}
        };
        \draw[arrow] (teacher) -- (teacher_out);
        
        % 5b. Student output
        \node[data, above=0.9cm of student, minimum width=2.3cm] (student_out) {
            \textbf{Logits}\\
            {\footnotesize (Student)}
        };
        \draw[arrow] (student) -- (student_out);
        
        % 6. Temperature scaling - optimized spacing
        \node[data, above=1.1cm of mapping, yshift=3.9cm, minimum width=3cm] (temp) {
            \textbf{Temperature Scaling}
        };
        \draw[arrow] (teacher_out) -| (temp.west);
        \draw[arrow] (student_out) -| (temp.east);
        
        % 7. Multi-task losses
        \node[loss, above left=0.9cm and 1.3cm of temp] (ce) {
            \textbf{Cross-Entropy}
        };
        \node[loss, above=0.9cm of temp] (kl) {
            \textbf{KL Divergence}
        };
        \node[loss, above right=0.9cm and 1.3cm of temp] (time_loss) {
            \textbf{Time Loss}
        };
        
        \draw[arrow] (temp) -- (kl);
        
        % Input → Cross-Entropy: Route left to CE south edge with label
        \draw[arrow, dashed] (input) -| node[pos=0.25, rotate=90, above, font=\footnotesize] {labels} (ce.south);
        
        % Logits → Time Loss: Route right to Time Loss east edge
        \draw[arrow, dashed] (student_out.east) -- ++(0.6,0) |- (time_loss.east);
        
        % 8. Total loss
        \node[loss, above=0.9cm of kl, minimum width=4.5cm, minimum height=0.8cm] (total) {
            \textbf{Total Loss}\\
            {\footnotesize Weighted Combination}
        };
        \draw[arrow] (ce) |- (total.west);
        \draw[arrow] (kl) -- (total);
        \draw[arrow] (time_loss) |- (total.east);
        
        % 9. Gradient flow (top)
        \draw[flowArrow, red!60, line width=0.9pt] (total) -- ++(0, 0.8) node[above, font=\normalsize\bfseries] {Backprop to Student};

    \end{tikzpicture}
    \caption{Knowledge distillation framework with bottom-to-top information flow. The frozen teacher LM-TAD provides soft targets via temperature-scaled distributions over grid cells. The trainable student HOSER learns from hard labels (cross-entropy), auxiliary tasks (time prediction), and soft teacher knowledge (KL divergence). The road-grid mapping bridges the cross-vocabulary gap, enabling knowledge transfer despite architectural and vocabulary differences.}
    \label{fig:distillation-framework}
\end{figure}


\subsection{Vocabulary Alignment}
\label{sec:method-vocab}

The cross-vocabulary mapping $\psi$ (\autoref{def:vocab-mapping}) enables knowledge transfer between the teacher's grid-based representation and the student's road-based representation. The mapping assigns each road segment to its containing grid cell based on centroid coordinates (detailed algorithm in \hyperref[app:vocab-mapping-alg]{Appendix~\ref*{app:vocab-mapping-alg}}).

\begin{remark}[Many-to-One Mapping]
    Multiple roads may correspond to a single grid cell, with higher density in urban cores.
\end{remark}

\begin{remark}[Output Space Alignment]
    The cross-vocabulary mapping introduces distributional differences between the teacher's grid-cell representation and the student's road-segment representation. Recent work on cross-vocabulary KD~\cite{zhangDualSpaceFrameworkGeneral2025} demonstrates that bridging distributions from different output spaces can limit teacher-student similarity. Our approach employs a fixed deterministic mapping $\psi$ based on spatial centroids, prioritizing computational simplicity and interpretability over alignment complexity.
\end{remark}

\subsubsection{Probability Renormalization}
\label{sec:method-renorm}

Since the teacher produces distributions over all grid cells $\mathcal{Z}$ while the student operates over candidate roads $\mathcal{C}_t$, we extract and renormalize:
\begin{equation}
    \tilde{q}_t(c) = q(\psi(c) \mid \mathbf{z}_{1:t}) \quad \forall c \in \mathcal{C}_t
    \label{eq:extract}
\end{equation}
\begin{equation}
    q_t^{(\tau)}(c) = \frac{\exp(\log \tilde{q}_t(c) / \tau)}{\sum_{c' \in \mathcal{C}_t} \exp(\log \tilde{q}_t(c') / \tau)}
    \label{eq:renorm}
\end{equation}

\subsection{Knowledge Distillation Framework}
\label{sec:method-kl}

\subsubsection{Temperature Scaling}

\begin{theorem}[Temperature-Scaled Knowledge Transfer]
    \label{thm:temp-scaling}
    Given teacher logits $\boldsymbol{\ell}^{\mathcal{L}} \in \mathbb{R}^{|\mathcal{C}|}$ and student logits $\boldsymbol{\ell}^{\mathcal{H}} \in \mathbb{R}^{|\mathcal{C}|}$, the distillation loss is:
    \begin{equation}
        \mathcal{L}_{\text{KL}}^{(\tau)} = \tau^2 \sum_{i \in \mathcal{C}_t} q_i^{(\tau)} \log \frac{q_i^{(\tau)}}{p_i^{(\tau)}}
        \label{eq:distill-loss}
    \end{equation}
    where:
    \begin{align}
        q_i^{(\tau)} & = \frac{\exp(\ell_i^{\mathcal{L}} / \tau)}{\sum_{j \in \mathcal{C}_t} \exp(\ell_j^{\mathcal{L}} / \tau)} \label{eq:teacher-dist} \\
        p_i^{(\tau)} & = \frac{\exp(\ell_i^{\mathcal{H}} / \tau)}{\sum_{j \in \mathcal{C}_t} \exp(\ell_j^{\mathcal{H}} / \tau)} \label{eq:student-dist}
    \end{align}
    The $\tau^2$ scaling factor preserves gradient magnitudes as $\tau$ increases.
\end{theorem}

\begin{proof}
    The gradient with respect to student logit $\ell_i^{\mathcal{H}}$ is:
    \begin{equation}
        \frac{\partial \mathcal{L}_{\text{KL}}}{\partial \ell_i^{\mathcal{H}}} = \frac{1}{\tau}(p_i^{(\tau)} - q_i^{(\tau)})
    \end{equation}
    which scales as $\mathcal{O}(1/\tau)$. Multiplying by $\tau^2$ ensures $\mathcal{O}(\tau)$ scaling, preventing gradient vanishing in the high-temperature regime where distillation is most effective~\cite{hintonDistillingKnowledgeNeural2015}.
\end{proof}

The reverse KL divergence has several desirable properties for knowledge distillation. It exhibits mode-seeking behavior, encouraging the student to learn specific behaviors from the teacher rather than spreading probability across multiple suboptimal options. Additionally, it reduces exposure bias by training the student on its own predicted distributions~\cite{OnPolicyDistillation}. Recent work demonstrates that this loss function enables dense, per-token supervision that is significantly more compute-efficient than sparse reward signals~\cite{OnPolicyDistillation}.

The role of temperature in distillation has been extensively studied. Higher temperatures smooth the output distribution, enabling transfer of relational information between classes~\cite{hintonDistillingKnowledgeNeural2015}. Recent work on LLM distillation emphasizes that temperature helps preserve diversity in the target distribution, preventing the narrowing effect that can occur with aggressive supervision~\cite{singhORPODistillMixedPolicyPreference2025}. The optimal temperature is task- and dataset-dependent, requiring empirical tuning for each application.

\paragraph{Hyperparameter Optimization}
We employ Optuna with the CMA-ES sampler to identify optimal distillation hyperparameters for each dataset. The search space spans $\lambda \in [0.001, 0.1]$ (log scale), $\tau \in [1.0, 5.0]$ (linear scale), and window size $w \in [2, 8]$ (integer). For the Beijing dataset, a 12-trial optimization identified optimal values:
\begin{itemize}[leftmargin=*,noitemsep]
    \item $\lambda = 0.0014$ (KL divergence weight)
    \item $\tau = 4.37$ (temperature)
    \item $w = 7$ (teacher context window, steps)
\end{itemize}
These hyperparameters achieve 57.24\% validation accuracy (marginal +0.01\% over vanilla's 57.23\%), yet produce dramatic generation quality improvements (85--89\% OD match rate vs vanilla's 12\%). This disconnect between validation and generation metrics indicates that next-step prediction accuracy is a poor proxy for long-horizon trajectory realism.

\textbf{Dataset-specific tuning is mandatory}---Porto optimal hyperparameters diverge significantly from Beijing: $\lambda = 0.00598$ (4.3$\times$ higher), $\tau = 2.515$ (42\% lower), $w = 4$ (43\% shorter). Cross-dataset hyperparameter transfer fails, requiring per-dataset HPO study.

\subsubsection{Combined Training Objective}

The total loss combines supervised learning with knowledge distillation:
\begin{equation}
    \mathcal{L}_{\text{total}} = \underbrace{\mathcal{L}_{\text{CE}}(p, y)}_{\text{hard targets}} +
    \alpha \underbrace{\mathcal{L}_{\text{time}}(\hat{t}, t)}_{\text{auxiliary}} +
    \lambda \underbrace{\mathcal{L}_{\text{KL}}^{(\tau)}(q^{(\tau)}, p^{(\tau)})}_{\text{soft targets}}
    \label{eq:total-loss}
\end{equation}
where $\alpha$ is fixed and $\lambda$ is tuned to balance supervised and distillation objectives.

\subsection{Training Pipeline}
\label{sec:method-training}

\begin{algorithm}[t]
    \caption{Knowledge Distillation Training}
    \label{alg:distillation}
    \begin{algorithmic}[1]
        \Require Dataset $\mathcal{D}$, Teacher $\mathcal{L}_\phi$, Student $\mathcal{H}_\theta$, hyperparameters $\{\lambda, \tau, w, \eta\}$
        \Ensure Trained parameters $\theta^*$
        \State Initialize $\theta$ from normal distribution
        \For{epoch $= 1$ to $E$}
        \For{$(\mathbf{r}, y) \in \mathcal{D}$}
        \State $\mathbf{z} \gets \psi(\mathbf{r})$ \Comment{Map roads to grid cells}
        \State $\mathbf{q} \gets \mathcal{L}_\phi(\mathbf{z}_{t-w:t})$ \Comment{Teacher inference (no gradient)}
        \State $q^{(\tau)} \gets \text{Renormalize}(\mathbf{q}, \mathcal{C}_t, \tau)$ \Comment{Eq.~\eqref{eq:renorm}}
        \State $\boldsymbol{\ell} \gets \mathcal{H}_\theta(\mathbf{r}, \mathcal{C}_t)$ \Comment{Student logits}
        \State $p^{(\tau)} \gets \text{Softmax}(\boldsymbol{\ell} / \tau)$
        \State $\mathcal{L} \gets \mathcal{L}_{\text{CE}}(\boldsymbol{\ell}, y) + \lambda \mathcal{L}_{\text{KL}}^{(\tau)}(q^{(\tau)}, p^{(\tau)})$
        \State $\theta \gets \theta - \eta \nabla_\theta \mathcal{L}$
        \EndFor
        \EndFor
        \State \Return $\theta^*$
    \end{algorithmic}
\end{algorithm}

The teacher parameters $\phi$ remain frozen throughout training. Per-iteration complexity is $\mathcal{O}(B \cdot t \cdot (V + k^2))$ where teacher inference dominates at $\mathcal{O}(B \cdot t \cdot V)$. Implementation details are provided in \autoref{sec:implementation}.

\subsection{Inference and Evaluation}
\label{sec:method-inference}

At inference time, we employ only the trained student $\mathcal{H}_{\theta^*}$ with beam search to generate trajectories. Given origin-destination pair $(r_o, r_d)$, the student produces complete trajectories efficiently.

We evaluate using global distribution metrics (Jensen-Shannon Divergence) and local trajectory metrics (Hausdorff, DTW, EDR) computed separately on training and test OD pairs. Full evaluation details are provided in \autoref{sec:eval-metrics}.

\section{Data and Preprocessing}
\label{sec:data-preprocessing}

\subsection{Dataset Description}
\label{sec:data}

The dataset used in this study consisted of Beijing taxi GPS data collected between 25.11.2019 and 01.12.2019. Each day contained approximately 16GB of raw GPS data, capturing the detailed movements of taxis throughout the metropolitan area. This large-scale dataset provided a rich source of real-world taxi routes for analysis and synthetic data generation.

\subsection{Raw Data Statistics and Characteristics}
\label{sec:raw-data}

\begin{compactoutline}
  \outlineitem{Dataset Scale and Volume -- Daily data volumes, GPS record counts, active vehicle numbers}
  \outlineitem{Temporal Sampling Characteristics -- Sampling rate variations, reconstruction challenges}
  \outlineitem{Trip Duration and Distance Analysis -- Statistical distributions, median values, range analysis}
  \outlineitem{Spatial Coverage Assessment -- Geographic extent, urban core concentration patterns}
  \outlineitem{Temporal Pattern Analysis -- Diurnal patterns, peak activity periods, weekend variations}
  \outlineitem{Anomaly Detection Context -- Normal behavior definitions across temporal contexts}
\end{compactoutline}

\subsection{Data Source and Privacy Issues}
\label{sec:data-source}

\begin{compactoutline}
  \outlineitem{Data Source Documentation -- Origin, collection methodology, licensing}
  \outlineitem{Privacy Risk Assessment -- Individual identification risks, sensitive locations}
  \outlineitem{Regulatory Compliance -- Legal requirements and constraints}
  \outlineitem{Pseudo-anonymisation Strategy -- Approach to privacy protection}
\end{compactoutline}

\subsection{Data Quality Assessment}
\label{sec:quality-assessment}

\begin{compactoutline}
  \outlineitem{Data Quality Issues Analysis -- Missing data, GPS accuracy, temporal gaps}
  \outlineitem{Anomalous Data Detection -- Identification of erroneous trajectories}
  \outlineitem{Exclusion Criteria -- Examples of excluded data with justifications}
  \outlineitem{Data Completeness Analysis -- Coverage and representativeness evaluation}
\end{compactoutline}

\subsection{Preprocessing Pipeline}
\label{sec:preprocessing}


\begin{compactoutline}
  \outlineitem{Data Cleaning Framework -- Error correction and outlier removal}
  \outlineitem{Trajectory Reconstruction -- Handling missing GPS points and temporal gaps}
  \outlineitem{DiffTraj-LM-TAD Format Compatibility -- Ensuring seamless data flow between generation and detection components}

  To ensure standardization and reproducibility, this research implements consistent data formatting protocols compatible with both DiffTraj and LM-TAD frameworks. The preprocessing pipeline standardizes GPS trajectory representations to support continuous coordinate processing for DiffTraj generation and discrete token sequences for LM-TAD anomaly detection, enabling seamless integration between synthetic generation and anomaly detection components.

  \outlineitem{Parameter Selection -- Preprocessing parameters with justifications}
  \outlineitem{Quality Control Measures -- Validation of preprocessing results}
\end{compactoutline}

\subsection{Processed Data Statistics}
\label{sec:processed-data}

\begin{compactoutline}
  \outlineitem{Post-Processing Statistics -- Comparison with raw data characteristics}
  \outlineitem{Quality Improvements -- Quantitative assessment of preprocessing impact}
  \outlineitem{Data Reduction Analysis -- Volume and coverage after preprocessing}
  \outlineitem{Validation Results -- Verification of data quality and integrity}
\end{compactoutline}


\section{Experimental Setup and Results}
\label{sec:evaluation}

The evaluation framework is designed to be comprehensive and multi-faceted, assessing the generated data from three critical perspectives: the performance of the anomaly detection system, the quality of the synthetic data, and the robustness of the privacy-preserving mechanisms.

\subsection{Experimental Design and Validation Strategy}
\label{sec:exp-design}

The experimental design implements a per-city pipeline validation strategy to assess framework generalizability and robustness across diverse urban environments.

\begin{description}
  \item[Primary Framework Development] The complete three-phase framework (baseline generation, anomaly mining, iterative refinement) is developed and optimized using the Beijing taxi dataset. This provides a comprehensive implementation baseline with well-characterized performance metrics.
  \item[Independent Pipeline Replication] To assess generalizability, the entire framework pipeline is independently executed on additional datasets from Chengdu and Xi'an. Each city receives separate model training, privacy budget allocation, and iterative refinement cycles, ensuring fair comparison across urban environments with different characteristics.
  \item[Cross-City Performance Analysis] Framework performance is systematically compared across cities using identical evaluation metrics, enabling assessment of: (1) baseline synthetic data quality across different urban contexts, (2) anomaly detection effectiveness in diverse mobility patterns, and (3) privacy protection consistency across varied trajectory characteristics.
  \item[Public vs. Private Dataset Validation] The framework includes comparative analysis between public and private datasets from Beijing to investigate potential discrepancies and biases in publicly available data, ensuring robust evaluation foundations.
\end{description}

\subsection{Anomaly Detection Performance}
\label{sec:results}

The performance of the anomaly detection system is evaluated using metrics appropriate for imbalanced datasets, where anomalies are rare.

\begin{description}
  \item[Key Performance Metrics] Evaluation focuses on Precision, Recall, and the F1-Score, which provide a balanced view of the detector's ability to correctly identify rare anomalous instances. These metrics are standard in the field and are used in comparable studies involving language model-based anomaly detection~\cite{mbuyaTrajectoryAnomalyDetection2024}.
  \item[AUC-ROC and AUC-PR] The Area Under the Receiver Operating Characteristic Curve (AUC-ROC) and the Area Under the Precision-Recall Curve (AUC-PR) are used to assess the model's overall discriminative power, with AUC-PR being particularly informative for imbalanced class distributions.
\end{description}

\subsection{Synthetic Data Quality Evaluation}
\label{sec:synthetic-eval}

The quality of the generated synthetic data is assessed using a standardized framework to ensure it is both realistic and useful for downstream tasks.

\begin{description}
  \item[Standardized Quality Assessment] The \textbf{SDMetrics} library is employed to systematically evaluate the synthetic data. This includes assessing statistical resemblance to real data (Resemblance), utility for machine learning tasks (Utility), and protection against disclosure (Privacy).
  \item[Statistical Fidelity] Distribution comparison tests (e.g., Kolmogorov-Smirnov, Jensen-Shannon divergence) are used to quantitatively measure the statistical similarity between the real and synthetic trajectory data distributions for key properties like trip duration and distance, an approach consistent with the evaluation in~\cite{zhuDiffTrajGeneratingGPS2023}.
  \item[Downstream Task Performance] The utility of the synthetic data is further validated by evaluating the performance of downstream models (e.g., travel time estimation, destination prediction, route classification) trained on the synthetic data versus models trained on real data. This evaluation approach is consistent with the utility assessment strategy in~\cite{zhuDiffTrajGeneratingGPS2023}, where synthetic data utility was demonstrated through downstream prediction tasks.
\end{description}

\subsection{Privacy Preservation Assessment}
\label{sec:privacy-eval}

The privacy guarantees of the synthetic data are evaluated through a series of attack simulations designed to test its resilience against re-identification.

\begin{description}
  \item[Membership Inference Attacks] Tests are conducted to determine whether an adversary can successfully identify whether a specific, real trajectory was part of the original training dataset used to create the synthetic data.
  \item[Trajectory Reconstruction Attacks] The framework is evaluated on its ability to prevent an adversary from reconstructing individual, real-world trajectories from the synthetic dataset.
  \item[Standardized Privacy Attack Evaluation using scikit-mobility] The privacy resilience of synthetic trajectories is systematically evaluated using the scikit-mobility framework's comprehensive attack suite. This includes location-based attacks (LocationAttack for general location inference, UniqueLocationAttack for unique location identification), frequency-based attacks (LocationFrequencyAttack, LocationProbabilityAttack, LocationProportionAttack), sequential pattern attacks (LocationSequenceAttack), temporal correlation attacks (LocationTimeAttack), and context-specific attacks (HomeWorkAttack for home/work location inference). Each attack class provides standardized \texttt{assess\_risk()} methods that quantify privacy vulnerabilities, enabling consistent and reproducible privacy evaluation across different synthetic data generation approaches.
  \item[Privacy-Utility Trade-off] A quantitative analysis is performed to measure the balance between the level of privacy protection achieved and the resulting utility of the data for anomaly detection research.
\end{description}

\subsection{Computational Performance Analysis}
\label{sec:performance}

\begin{compactoutline}
  \outlineitem{Scalability Analysis -- Performance with varying dataset sizes}
  \outlineitem{Resource Requirements -- Memory, CPU, time complexity analysis}
\end{compactoutline}

\subsection{Ablation Study}
\label{sec:ablation}

To quantify the impact of key components in the framework, we plan to perform two ablation experiments:
\begin{itemize}
  \item \textbf{No Rule-Based Curation:} We expect that removing the rule-based curation step and relying solely on the unsupervised detector will lead to a drop in anomaly detection F1-score (e.g., from 0.81 to 0.74), and the proportion of interpretable anomalies is expected to decrease significantly (e.g., from 92\% to 61\%).
  \item \textbf{No Iterative Refinement:} We anticipate that using a single-pass (non-iterative) approach will reduce the number of unique anomaly categories generated (e.g., from 6 to 3), and the overall anomaly detection F1-score is expected to drop (e.g., from 0.81 to 0.76).
\end{itemize}
These anticipated results would demonstrate that both rule-based curation and iterative refinement are critical for achieving high anomaly interpretability, diversity, and detection performance.


\section{Conclusion and Future Work}
\label{sec:conclusion}

\subsection{Research Contributions Summary}

\begin{compactoutline}
  \outlineitem{Primary Contributions -- Novel synthetic generation framework, privacy-preserving anomaly detection}
  \outlineitem{Integrated Framework -- Novel integration of DiffTraj generation with LM-TAD anomaly detection for privacy-preserving trajectory research}
  \outlineitem{Cross-City Validation -- Demonstrated framework generalizability across diverse urban environments (Beijing, Chengdu, Xi'an)}
\end{compactoutline}

\subsection{Research Impact and Applications}

\begin{compactoutline}
  \outlineitem{Academic Impact -- Novel DiffTraj-LM-TAD integration methodology, privacy-preserving trajectory research advancement}
  \outlineitem{Practical Applications -- Urban transportation anomaly detection, ride-sharing route optimization, taxi fleet management}
  \outlineitem{Research Enablement -- Privacy-compliant datasets for trajectory analysis, reproducible anomaly detection evaluation frameworks}
\end{compactoutline}

\subsection{Limitations and Challenges}

\begin{compactoutline}
  \outlineitem{Current Limitations -- Computational complexity of iterative refinement, dependency on manual rule-based curation}
  \outlineitem{Technical Challenges -- Privacy-utility trade-offs, balancing anomaly diversity with generation quality}
  \outlineitem{Implementation Challenges -- DiffTraj-LM-TAD integration complexity, parameter sensitivity across different urban contexts}
\end{compactoutline}

\subsection{Future Research Directions}

\begin{compactoutline}
  \outlineitem{Methodological Extensions -- Automated rule-based curation, conditional generation refinements, multi-modal trajectory data integration}
  \outlineitem{Privacy Enhancement -- Advanced differential privacy mechanisms, federated learning integration for cross-city collaboration}
  \outlineitem{Evaluation Framework Extensions -- Long-term temporal pattern validation, real-time anomaly detection deployment, large-scale urban network evaluation}
\end{compactoutline}

\subsection{Concluding Remarks}

\skeletontext{Summary of the research significance, implications for urban transportation research, and the potential for practical deployment of privacy-preserving trajectory anomaly detection systems.}


\newpage

% ===== Bibliography =====
\bibliographystyle{splncs04}
\bibliography{references_new}

% ===== Appendix =====
\appendix
\section{Appendix}
\label{sec:appendix}

\subsection{Appendix Section}
\label{sec:appendix-section}

\subsection{Appendix Section}
\label{sec:appendix-section-2}

\end{document}